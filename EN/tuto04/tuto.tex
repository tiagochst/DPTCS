% -%-%-%-%-%-%-%-%-%-%-%-%-%-%-%-%-%-%-%-%-%-%-%-%-%
% ANG: Anglais % 
% Mentoring 2 %
% Data:09/12/2011                                 %
% Paris,France % 
% Groupe: %
% - Tiago Chedraoui Silva % 
% -%-%-%-%-%-%-%-%-%-%-%-%-%-%-%-%-%-%-%-%-%-%-%-%-%

\documentclass[a4paper,12pt]{article}

\usepackage[english,listings,algo]{tcs}

% Cover %
\def \ttprofname{Jan TABET} % teachers name
\def \ttabrv{ANG206} % abbreviation of names class
\def \ttabrvxt{P2} % period
\def \mytitle{The Moving Finger} % Big title
\def \mysubtitle{Tutorial English class - First Semester of 2012 } % subtitle
\def \ttauth{Tiago Chedraoui Silva} % author's name
\def \ttxt{Pigeonhole: 214 } % Extra text right side of name
\def \ttdate{February 02, 2012} % date

\begin{document}
\titleTMB 
\newpage
\spc{2} % double spacing


\section*{Book: The moving finger}
% \subsection*{Description}
\subsection*{Questions}

\begin{description}

\item[Who is sending anonumous letters? And why?]
  {
    Mr. Symmington was the responsible  for the anonymous letters. He planned to
    kill his  wife, and these  letters were  a way to  disguise he was  the real
    killer. So, he wrote all the letters in advance and started delivering them, until he could kill his wife.
  }
  

%\item[Why Lymstosk was the place chosen for it?]
%  {
%  }

  
\item[When did the anonymous letters start?]
  {
    In  the beginning  of the  book the  main character  received  one anonymous
    letter, which, talking later to the doctor of the city he discovered that it
    wasn't the first  letter received by people in Lymstosk,  which is the small
    country city when the story takes place.
  }
  
  
\item[Did Mrs. Symmington really commited suiced? Or was she killed by someone?]
  {
    Mrs. Symmington death  was because she drank cyanide,  which could have been
    put by someone  or by herself. In  the first time, we're lead  to think that
    she committed  suicide because of  an anonymous letter received,  which said
    that she had an affair outside of  her marriage which resulted in a baby. We
    are induced to believe that it would be true and she decided to take out her
    life, or that it wasn't true but the husband would not believe.

    Later, in the  book we discovered that she was killed  by her husband, which
    indeed was  the anonymous writer  of the letters.  He created a  death which
    seemed incidental.  To foil the police,  he copied the  letters from another
    case which had a woman anonymous writer, so when the police saw the letters
    they suspected to be a woman writer and not a man.

    Indeed  he put  cyanide  in the  top  cachet of  the ones  she  took in  the
    afternoon, and when she was dead he put cyanide in a cup of water, inducing
    us to think  that she drank cyanide.  He left a scrap of  paper written with
    his wife's handwriting 'I can't go on',  which was cut from a note she wrote
    a long time before.

  }

  
\item[How the mystery was solved?]
  {
    Later, a  new letter was  sent to  Elsie Holland, who  was in charge  of the
    kids. The letter said that she must got out of the house and was warning her
    that all  the city was  laughing at her,  because they knew she  wanted to
    marry the husband of  the dead woman. In the end of  the letter, there was a
    menace remembering what happened to the other girl, who was killed.
    
    The police man Nash,  was sure that the letter was sent by
    another  woman in the  house, more  specifically the  nurse of  the family,
    because he saw her typing the letter, but she denied everything.
    
    However, proofs were against her, in the cupboard under the stairs of the house, where no one
    normally look into, there was found pages
    of  an old  book, from  where  the letters  of anonymous  letters have  been
    cut. Also, a heavy pestle linked with the murder of house's maid disappeared
    from the doctor's house, who was brother of the accused girl.
    
    Nevertheless, they  got wrong woman! Indeed,  the police suspected  it was a
    woman regarding  the kind of anonymous  letters, but if  this letters didn't
    have something  related to the suicide, what  if it was disguise  to make us
    think that it was a suicide, while it was murder? 
    
    Miss Marple gave the clue, there  wasn't anonymous letters. If they were all
    fake, why  they were created?  If we took  out the letters from  the problem
    what we got? I woman was dead, I  who wanted to kill her? Her husband, so he
    could had his attractive young governess after his wife's death.
  }
  

%\item[Why does the doctor ignore Joanna?]
%  {
%  }

%\item[Which is the killers pattern?]
%  {
%  }

\item[Why is Agnes, one of the maids, killed a week after Mrs. Symmington suicide?]
% page 130
  {
    In the afternoon  of Mrs. Symmington suicide, Agnes has  returned  to home
    earlier because she quarreled violently with  her boyfriend in the middle of a
    walk.  Agnes expected that  her boyfriend  would return  to apologize,  so she
    waited in the kitchen, watching the front of the house through the window. 
    The police  thought that  she discovered who  was the anonymous  writer, who
    had  delivered the  letters by  himself.  As  she knew  the identity  of the
    killer, she was in a great danger. So, she was the next target.

    However, later the  police discover that there was no  deliverer, and as she
    did  not saw  anyone  delivering the  letters,  she knew  there  was not  an
    anonymous writer, which will lead us easily to the real killer.
  }

  
\end{description}



\subsection*{Recommendation}
Dear Aline,

You know  how much a love  Agatha Christie's books and  I read a  new one: ``The
moving finger'' which is considered to be one of the ten best books written by her. 
This is a first person narrated story, which sometimes is very nice, but this
time I felt a little annoyed. The narrator was not from police, indeed, he
tries to discover the mystery, but not all the time, which made me feel a little
bored, because in  a book of 250  pages about 70 were useless.  This pages were
giving time to things happen, and, for me, reading vague happenings aren't so nice.

However, this  was another Agatha's masterpiece. She  brightly linked everything
yin the book in a way that all its happenings were part of the solution, even the
most detailed one. The murder was very well prepared, what makes the solution even more interesting!

Another  thing, the  detective of  the book  is Miss  Marple, it  was  the first
appearance  of this  character in  Agatha's books,  however, she  almost  do not
participate in the story, but she has a crucial role in solving the mystery.

So, you  should wait for  a brightly  detective story, which  is a sort  of Miss
Marple cases,  brightly but  not the  best type of  books to  read. If  you love
detective stories, you should get it.   If you never read any Agatha's detective
stories I shall recommend first `` And then they were none'', also called ``Ten
little niggers'',  and ``The  crooked house'' if  you don't like  mysteries with
detectives. Or  a book  with the  detective called Hercule  Poirot, which  in my
opinion is better than Miss Marple, as he leads you to peculiar thoughts. 

Good reading, see you!
Tiago
\end{document}