% -%-%-%-%-%-%-%-%-%-%-%-%-%-%-%-%-%-%-%-%-%-%-%-%-%
% Tutorat 01 EN                                    %  
% Data:02/11/2011                                  %
% Paris,France                                     %
% - Tiago Chedraoui Silva                          %
% -%-%-%-%-%-%-%-%-%-%-%-%-%-%-%-%-%-%-%-%-%-%-%-%-%
\documentclass[dvips,11pt,xcolor=dvipsnames]{beamer}

%%% fontes %%%
\usepackage[french]{babel}
\usepackage[T1]{fontenc}
\usepackage[utf8]{inputenc}   
\usepackage{ae,aecompl,aeguill}  % pdfs plus beaux =)

%%% matematicos %%%
\usepackage{amsmath}
\usepackage{amssymb}
\usepackage{mathptmx}

%%% figuras %%%
\usepackage{graphicx}
\usepackage{wrapfig}

%%% tabelas %%%
\usepackage{colortbl}
\usepackage{array}
\usepackage{longtable}
\usepackage{fancyvrb}
\usepackage{color}

%%% outros %%%
\usepackage{url}
\usepackage{textcomp}
\usepackage{hyperref} %internal links
\usepackage{color}       
\usepackage{indentfirst} %retira padrao americano de paragrafos
\usepackage{multicol}    
\numberwithin{table}{section}
\numberwithin{figure}{section} %numercao de figuras por secao

%%% extras %%%
\RequirePackage{marvosym} % figuras \Letter \Email 
\usepackage{fancyhdr}     % Headers
\usepackage{epsf}
\usepackage{tikz}
\usetikzlibrary{arrows}
\tikzstyle{block}=[draw opacity=0.7,line width=1.4cm]

%%% Beamer style %%%
\usetheme{Warsaw}
%\usecolortheme[named=Green]{structure}
\usefonttheme[onlylarge]{structurebold} % fontes em negrito
\setbeamerfont*{frametitle}{size=\normalsize,series=\bfseries}%define tamanhos de letras
\setbeamertemplate{navigation symbols}{}% barra de navegacao superior
% \setbeamercovered{transparent}

% -%-%-%-%-%-%-%-%-%
% Inicio Slides  %
% -%-%-%-%-%-%-%-%-%

\title{Mentoring}                               
% Refletindo os valores humanos na era digital
\subtitle{A new ally against Cancer} 
\author[Tiago S.]{
  Tiago Chedraoui Silva \\
}
\institute{}

\date{November 03, 2011}
%\date{\today}

\begin{document}

\begin{frame}
  \titlepage
\end{frame}

%\begin{frame}{Outline}
%  \tableofcontents
%\end{frame}


\section{Introduction}

\begin{frame}{Cancer}
 
 \begin{block}{Three main therapies}
    \begin{itemize}
    \item Surgery 
    \item Chemotherapy 
    \item Radiation 
    \end{itemize}
  \end{block}

\pause \begin{alertblock}{More clearly}
    \begin{itemize}
    \item Slash 
    \item Poison 
    \item Burn 
    \end{itemize}
  \end{alertblock}

\end{frame}


%\begin{frame}{Outils}

%\end{frame}

\begin{frame}{How to attack the cancer?}
 \begin{block}{Idea}
    \begin{itemize}
    \item Prod body's own immune system to do a better job of fighting malignancies.
    \end{itemize}
  \end{block}

\pause \begin{block}{How?}
    \begin{itemize}
    \item Vaccines: FDA approved the first vaccine to treat cancer
    \end{itemize}
  \end{block}
\end{frame}


\section{A New Ally}

\begin{frame}{The vaccine}
 \begin{block}{Common vaccine}
    \begin{itemize}
    \item Prevents infections.
      \item Triggers a simple antibody response.
    \end{itemize}
  \end{block}

\pause  \begin{block}{Cancer vaccine}
    \begin{itemize}
    \item  Train the body  to recognize  and destroy  cancer cells  that already
      exists;
    \item After it continues killing malignant cells;
    \end{itemize}
  \end{block}

\pause \begin{alertblock}{Why different?}
    \begin{itemize}
    \item Antibodies response are not strong to kill cancer cells.
    \item Immune system needs T cells.
    \end{itemize}
  \end{alertblock}
\end{frame}




\begin{frame}{The T cells}
 \begin{exampleblock}{Two types of T cells}
    \begin{itemize}
    \item CD4 - Give orders about who and what to attack.
      \item CD8 - Destroy malignant cells.
    \end{itemize}
  \end{exampleblock}

\pause \begin{block}{What we need to know}
    \begin{itemize}
    \item As tumor  grows, it releases more substances  that actively suppress T
      cells;
    \item So, we need to treat early as possible.
    \item Vaccines with only CD8 cells had shown no benefit.
    \end{itemize}
  \end{block}

\pause \begin{alertblock}{A new vaccine}
    \begin{itemize}
    \item A mix of CD4 and CD8.
    \item Took T  cells from a patient  and train them to target  and attack the
      tumor;
    \item Injected them into the patient;
    \end{itemize}
  \end{alertblock}
\end{frame}



\section{Strategies}

\begin{frame}{Making a vaccine: three elements}
 \begin{exampleblock}{What?}
    \begin{itemize}
    \item What molecular feature in a  malignant tumor should  be recognized as
      foreign and target to be killed;
    \end{itemize}
  \end{exampleblock}

\pause \begin{exampleblock}{How?}
    \begin{itemize}
    \item How to deliver a trigger agent or vaccine to the immune system.
    \end{itemize}
  \end{exampleblock}

\pause \begin{exampleblock}{Who and When?}
    \begin{itemize}
    \item Which cancer patients to treat; 
      \item When during the course of their
      disease to administer the vaccine.
    \end{itemize}
  \end{exampleblock}
\end{frame}

\begin{frame}{Making a vaccine: three elements}
 \begin{block}{What?}
    \begin{itemize}
    \item Proteins (peptides);
    \item Difficult: genetic alterations of cancer cells;
    \item So, we need various examples of these peptides!
    \end{itemize}
  \end{block}

  \pause \begin{block}{How?}
    \begin{itemize}
    \item Other idea  is to use a  dendritic cells which alert the  T cells that
      something is wrong;
    \item Problem: T cells and dendritic must be genetically identical;
    \item We need to harvest them from each individual patient (\$\$)
    \end{itemize}
  \end{block}

  \pause \begin{alertblock}{When?}
    \begin{itemize}
    \item It may take a year  after the treatment for immune system start making
      substantial progress
    \end{itemize}
  \end{alertblock}
\end{frame}


\section{Conclusion}


\begin{frame}{Conclusion}
 \begin{block}{When}
    \begin{itemize}
    \item Earlier stages: vaccines.
    \item Advanced stages: vaccines+conventional treatment.
    \end{itemize}
  \end{block}

 \begin{block}{The future?}
    \begin{itemize}
    \item We can not make too many promises;
    \item Vaccine will have a prominent role over the next decade.
    \end{itemize}
  \end{block}

\end{frame}
%\section{Vocabulaire}
%\section{Écrire}
%\section{Culture Française}
%\section{Littérature Française}
%\section{Site intéressant}
\end{document}
