% -%-%-%-%-%-%-%-%-%-%-%-%-%-%-%-%-%-%-%-%-%-%-%-%-%
% ANG: Anglais % 
% Mentoring 2 %
% Data:09/12/2011                                 %
% Paris,France % 
% Groupe: %
% - Tiago Chedraoui Silva % 
% -%-%-%-%-%-%-%-%-%-%-%-%-%-%-%-%-%-%-%-%-%-%-%-%-%

\documentclass[a4paper,12pt]{article}

\usepackage[english,listings,algo]{tcs}

% Cover %
\def \ttprofname{Penelope POULTON} % teachers name
\def \ttabrv{ANG206} % abbreviation of names class
\def \ttabrvxt{P2 } % period
\def \mytitle{Preparations for 2014 World Cup and 2016 Olympic games} % Big title
\def \mysubtitle{Tutorial english class - Second Semester of 2011 } % subtitle
\def \ttauth{Tiago Chedraoui Silva} % author's name
\def \ttxt{Pigeonhole: 214 } % Extra text right side of name
\def \ttdate{December 8, 2011} % date

\begin{document}
\titleTMB 
\newpage
\spc{2} % double spacing

\section{Can great events be realized in Brazil?}

% Intro
% Which events will be staged in Brazil ? And, specifically, in which cities ?
% What does the press think about, Brazil, Brazil's infrastructure? 

In October 2007, Fédération Internationale de Football Association (FIFA), the
sport's global governing body, chose Brazil as the host of 2014 world's cup.
And, two years after, in October 2009, International Olympic Committee (IOC)
announced  that Rio de Janeiro,  will stage  the 2016  Olympic  and Paralympics
Games.  
Although its beautiful landscapes and hospitable people,
Brazil  is seen  by press  as a  doubtful country  regarding  infrastructure and
security.

% Why a doubtful country?
On the security side, I think the press, based on various data, can strongly doubt my country;
because, although Brazil holds the seventh economy in the world, its HDI (Human
Development Index) is only the eighty-fourth; which means we have some problems.
But, is this  classification acceptable? Unfortunately, in a  country where child
labor still exists, where illiteracy numbers
reaches more  than five percent of  population and  where crime scares
population, this  classification should  be said as  correct. Also, I'm  a great
believer that security is one of  the biggest and oldest problems of Brazil, and
will not be solved in a short period of time.


On infrastructure side, the imposed deadline to begin works related to
world cup was may 2010, but at that time, not even fifty percent had begun, this
delay made the press believe in a certain brazilian disability to hold the World
Cup. A year later, all stadiums have already begun, but not all airports. 
This scenario  provides a  prevision that some  airports will not  be completed,
which means that people will be able to travel, however some problems as lack of
flights and delays should be seen.

On the  other hand,  if Brazil  will suffer in  the World  Cup, Rio  de Janeiro,
Brazil's second  biggest city , will not  only meet all the  standards, but also
will show why it is a spectacular city, being considered as one of the best Brazilian tourism cities.
The first  difference for this totally opposite scenario is that Rio de
Janeiro is a city, so in terms of size, less infrastructure is necessary, and
as Rio already has  a good base, only a small expansion  will be performed.  The
greatest problem  was the Crime, however  the government began  to firmly combat
crime  and investing  in some  poor  areas, with  the purpose  to rise  people's
condition of life and take people out of drugs market, which had been intense in the slums for a long time.


Finally, I think world cup will develop  Brazil, and Rio de Janeiro will be more
beautiful as  never; I hope that  everything constructed will be  used after and
not destroyed. For example, the training centers should be used for future athletes,
our children, independently of their  wealth conditions. In my view, Brazil will
show everyone that money is not the most important thing in life, but love is. I will
be waiting for a warm reception and a lot of happiness in these two events.

%specified, giving a high comfort to everyone capable to pay a fortune in tickets

%this data  is correct, however Brazil  is the world's fifth  largest country and
%which  part have  a  different situation,  southeast  and south  parts are  more
%developed. 

% Infrastructure
%I  firmly believe  that  Brazil is  not only  going  to meet  all the  standards
%specified, giving a high comfort to everyone capable to pay a fortune in tickets,
%but also it will show  that luxury is the less important  in life, as  show by
%south africans, money will not bring us happy;

%we should see more sun sets, complain less and live more.

% To do:
% Explain my country, and it difficulties; 

% Why is Brazil capable to handle with this?
% How was the scenario and how is it today?
% Compare with South Africa
% Show as example Itaquera,
% Compare time with Dubai

% I firmly believe
% I'm a great believer
% I feel
% I think
% I can't understand
% Contrary to popular believe
% In my view/ To my mind
% From my point of view

\end{document}