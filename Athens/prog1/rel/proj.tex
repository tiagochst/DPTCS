% -%-%-%-%-%-%-%-%-%-%-%-%-%-%-%-%-%-%-%-%-%-%-%-%-%
% INF226 % 
% Data:22/03/2012                                 %
% Paris,France                                    % 
% Groupe:                                         %
% - Tiago Chedraoui Silva                   % 
% -%-%-%-%-%-%-%-%-%-%-%-%-%-%-%-%-%-%-%-%-%-%-%-%-%

% enviar para irene.charon@telecom-paristech.fr

\documentclass[a4paper,11pt]{article}

\usepackage[francais,listings,algo]{tcs}


% Cover %
\def \ttprofname{Olivier Hudry} % teachers name
\def \ttabrv{INF226} % abbreviation of names class
\def \ttabrvxt{} % period
\def \mytitle{Problème du voyager de commerce} % Big title
\def \mysubtitle{Travaux Pratique 1 - Première semestre de 2012} % subtitle
\def \ttauthi{Tiago CHEDRAUOI SILVA} % author's name
\def \ttxti{Casier: 214 } % Extra text right side of name
\def \ttdate{Mars 22, 2012} % date

\begin{document}
\titleTMB 
\newpage
%\tableofcontents
%\listoffigures
%\newpage

\section*{Matériel}
Pour le travaux pratique on a utilise un ordinateur un peut lent:
\begin{itemize}
\item Intel(R) Core(TM)2 Duo CPU T5250 @ 1.50GHz
\item cache size : 2048 KB
\item Ram: 3Gb
\end{itemize}

\section*{Changements}
On a inclue  la bibliothèque ``math.h'' et on a changé  le Makefile pour inclure
le flag -lm necessaire pour compiler cette bibliothèque.

\section*{Partie 1}

\begin{table}[h!]

\caption{Nombre de noeuds visités sans sans utiliser la relaxation lagrangienne. }
\begin{centering}
\begin{tabular}{|c|c|c|c|}
\hline 
Graphe & Nombre de noeuds & Temps & Poids\tabularnewline
\hline 
\hline 
g5 & 8 & 0.000000 secondes  & 30\tabularnewline
\hline 
g10 & 511 & 0.000000 secondes  & 35\tabularnewline
\hline 
g15 & 9980 & 0.028995 secondes  & 20\tabularnewline
\hline 
g20 & 35137 & 0.158975 secondes  & 52\tabularnewline
\hline 
g22 & 23438 & 0.121981 secondes  & 120\tabularnewline
\hline 
g23 & 528589 & 2.809572 secondes & 269\tabularnewline
\hline 
g25 & 3292515 & 20.954814 secondes & 237\tabularnewline
\hline 
g30 & Temps non raisonnables & - & -\tabularnewline
\hline 
g50 & Temps non raisonnables & - & -\tabularnewline
\hline 
g100 & Temps non raisonnables & - & -\tabularnewline
\hline 
\end{tabular}
\par\end{centering}

\end{table}


\section*{Partie 2}

Le code pour ces calculs est au dessous.
\begin{multicols}{2}
  \lstinputlisting[title=\textbf{Relaxation}]{relaxation.c}
\end{multicols}


\section*{Partie 3}

En utilisant le programme de relaxation, on voit que le temps nécessaire
pour trouver la bonne réponse a  diminué de façon significative. Par exemple, le
graphe g30,g50 et g100 ont donné  une réponse en quelques secondes, cela n'était
pas possible avant.

\begin{table}[h!]
\begin{centering}
\begin{tabular}{|c|c|c|c|}
\hline 
Graphe & Nombre de noeuds & Temps & Poids\tabularnewline
\hline 
\hline 
g5 & 4 & 0.000000 secondes  & 30\tabularnewline
\hline 
g10 & 36 & 0.000999 secondes & 35\tabularnewline
\hline 
g15 & 96 & 0.002999 secondes & 20\tabularnewline
\hline 
g20 & 71 & 0.004999 secondes & 52\tabularnewline
\hline 
g22 & 59 & 0.003999 secondes & 120\tabularnewline
\hline 
g23 & 301 & 0.015997 secondes  & 269\tabularnewline
\hline 
g25 & 234 & 0.015997 secondes  & 237\tabularnewline
\hline 
g30 & 514 & 0.048992 secondes & 212\tabularnewline
\hline 
g50 & 4583 & 1.046840 secondes  & 214\tabularnewline
\hline 
g100 & 54262  & 46.290962 secondes  & 283\tabularnewline
\hline 
\end{tabular}
\par\end{centering}

\caption{Nombre de noeuds visités avec la relaxation lagrangienne (alpha =
1.5 nombre interactions = 10 borne = 5000 gama=2). }
\end{table}


\newpage
Après voir  que la relaxation a amélioré  la vitesse, on doit  chercher pour les
meilleurs paramètres pour notre relaxation.
Ainsi, on a changé l'alpha (pas), le nombre itérations et la valeur de la borne.


\begin{table}[h!]
\begin{centering}
\begin{tabular}{|c|c|}
\hline 
Alpha & Temps\tabularnewline
\hline 
\hline 
0.1 & 5.158214 secondes\tabularnewline
\hline 
0.3 & 3.011542 secondes\tabularnewline
\hline 
0.5 & 1.501771 secondes\tabularnewline
\hline 
0.75 & 1.334797 secondes\tabularnewline
\hline 
1 & 1.197817 secondes\tabularnewline
\hline 
1.3 & 1.146825 secondes\tabularnewline
\hline 
1.4 & 1.134827 secondes\tabularnewline
\hline 
1.5 & 1.038842 secondes\tabularnewline
\hline 
1.6 & 1.123829 secondes\tabularnewline
\hline 
1.8 & 1.087834 secondes\tabularnewline
\hline 
2 & 1.334797 secondes\tabularnewline
\hline 
2.5 & 1.309800 secondes\tabularnewline
\hline 
3 & 1.485774 secondes\tabularnewline
\hline 
\end{tabular}
\par\end{centering}

\caption{Comparaison alpha: Graphe g50 :Nombre de noeuds visités avec la relaxation
lagrangienne (nombre interactions = 10 borne = 5000 gama=2). }
\end{table}

Pour l'alpha, on a trouve: $Alpha = 1.5$.

\begin{table}[h!]
\begin{centering}
\begin{tabular}{|c|c|}
\hline 
Nb iteration & Temps\tabularnewline
\hline 
\hline 
2 & 3.753429 secondes\tabularnewline
\hline 
3 & 2.238659 secondes\tabularnewline
\hline 
4 & 1.715739 secondes \tabularnewline
\hline 
5 & 1.467776 secondes\tabularnewline
\hline 
10 & 1.032842 secondes\tabularnewline
\hline 
12 & 1.087834 secondes\tabularnewline
\hline 
15 & 0.981850 secondes\tabularnewline
\hline 
18 & 1.432782 secondes \tabularnewline
\hline 
20 & 1.126828 secondes\tabularnewline
\hline 
25 & 1.495772 secondes\tabularnewline
\hline 
50 & 1.861716 secondes\tabularnewline
\hline 
100 & 1.781729 secondes\tabularnewline
\hline 
\end{tabular}
\par\end{centering}

\caption{Comparaison nb iteration: Graphe g50 :Nombre de noeuds visités avec
la relaxation lagrangienne (nombre interactions = 10 borne = 5000
gama=2). }
\end{table}


Pour le nombre d'interaction, on a trouve: $interation = 15$.



\begin{table}[h!]
\begin{centering}
\begin{tabular}{|c|c|}
\hline 
Borne & Temps\tabularnewline
\hline 
\hline 
300 & 0.820875 secondes\tabularnewline
\hline 
500 & 0.720891 secondes\tabularnewline
\hline 
800 & 1.254809 secondes\tabularnewline
\hline 
1000 & 1.071837 secondes\tabularnewline
\hline 
2000 & 1.188819 secondes\tabularnewline
\hline 
3000 & 1.153824 secondes\tabularnewline
\hline 
5000 & 0.990849 secondes\tabularnewline
\hline 
10000 & 0.953854 secondes\tabularnewline
\hline 
20000 & 1.417784 secondes\tabularnewline
\hline 
100000 & 1.083835 secondes\tabularnewline
\hline 
\end{tabular}
\par\end{centering}

\caption{Comparaison des bornes: Graphe g50 :Nombre de noeuds visités avec
la relaxation lagrangienne (alpha = 1.5 nombre interactions = 15 gama=2). }
\end{table}


Pour la valeur de la borne, on a trouve: $borne = 500$.

En choisissant les meilleurs paramètres on a ajoute le code suivante, pour changer $\gamma$:
\begin{itemize}
\item norme = pow(norme,gama);
\end{itemize}

En choisissant un $\gamma$ égal a 1 est très lent.



\begin{table}[h!]
\begin{centering}
\begin{tabular}{|c|c|}
\hline 
$\gamma$ & Temps\tabularnewline
\hline 
\hline 
1 & Temps non raisonnables\tabularnewline
\hline 
1.5 & Temps non raisonnables\tabularnewline
\hline 
2 & 0.707892 secondes\tabularnewline
\hline 
2.5 & 2.824570 secondes\tabularnewline
\hline 
3 & 0.706892 secondes\tabularnewline
\hline 
4 & 10.137458 secondes\tabularnewline
\hline 
\end{tabular}
\par\end{centering}

\caption{Comparaison des $\gamma$ - Graphe g50 :alpha = 1.5 nombre interactions = 15 borne=500. }
\end{table}

\end{document}

