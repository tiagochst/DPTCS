%% LyX 2.0.2 created this file.  For more info, see http://www.lyx.org/.
%% Do not edit unless you really know what you are doing.
\documentclass[english]{article}
\usepackage[T1]{fontenc}
\usepackage[latin9]{inputenc}

\makeatletter

%%%%%%%%%%%%%%%%%%%%%%%%%%%%%% LyX specific LaTeX commands.
%% Because html converters don't know tabularnewline
\providecommand{\tabularnewline}{\\}

\@ifundefined{showcaptionsetup}{}{%
 \PassOptionsToPackage{caption=false}{subfig}}
\usepackage{subfig}
\makeatother

\usepackage{babel}
\begin{document}
\begin{table}
\begin{centering}
\begin{tabular}{|c|c|c|c|}
\hline 
Graphe & Nombre de noeuds & Temps & Poids\tabularnewline
\hline 
\hline 
g5 & 8 & 0.000000 secondes  & 30\tabularnewline
\hline 
g10 & 511 & 0.000000 secondes  & 35\tabularnewline
\hline 
g15 & 9980 & 0.028995 secondes  & 20\tabularnewline
\hline 
g20 & 35137 & 0.158975 secondes  & 52\tabularnewline
\hline 
g22 & 23438 & 0.121981 secondes  & 120\tabularnewline
\hline 
g23 & 528589 & 2.809572 secondes & 269\tabularnewline
\hline 
g25 & 3292515 & 20.954814 secondes & 237\tabularnewline
\hline 
g30 & Temps non raisonnables & - & -\tabularnewline
\hline 
g50 & Temps non raisonnables & - & -\tabularnewline
\hline 
g100 & Temps non raisonnables & - & -\tabularnewline
\hline 
\end{tabular}
\par\end{centering}

\caption{Nombre de noeuds visit�s sans sans utiliser la relaxation lagrangienne. }


\end{table}


\begin{table}
\begin{centering}
\begin{tabular}{|c|c|c|c|}
\hline 
Graphe & Nombre de noeuds & Temps & Poids\tabularnewline
\hline 
\hline 
g5 & 4 & 0.000000 secondes  & 30\tabularnewline
\hline 
g10 & 36 & 0.000999 secondes & 35\tabularnewline
\hline 
g15 & 96 & 0.002999 secondes & 20\tabularnewline
\hline 
g20 & 71 & 0.004999 secondes & 52\tabularnewline
\hline 
g22 & 59 & 0.003999 secondes & 120\tabularnewline
\hline 
g23 & 301 & 0.015997 secondes  & 269\tabularnewline
\hline 
g25 & 234 & 0.015997 secondes  & 237\tabularnewline
\hline 
g30 & 514 & 0.048992 secondes & 212\tabularnewline
\hline 
g50 & 4583 & 1.046840 secondes  & 214\tabularnewline
\hline 
g100 & 54262  & 46.290962 secondes  & 283\tabularnewline
\hline 
\end{tabular}
\par\end{centering}

\caption{Nombre de noeuds visit�s avec la relaxation lagrangienne (alpha =
1.5 nombre interactions = 10 borne = 5000 gama=2). }
\end{table}


\begin{table}
\begin{centering}
\subfloat[Alpha=0.5]{\begin{centering}
\begin{tabular}{|c|c|c|}
\hline 
Graphe & Nb noeuds & Temps\tabularnewline
\hline 
\hline 
g50 & 7301 & 1.497772 secondes\tabularnewline
\hline 
g100 & 31815 & 34.527750 secondes \tabularnewline
\hline 
\end{tabular}
\par\end{centering}



}\subfloat[Alpha=2.0]{\begin{centering}
\begin{tabular}{|c|c|c|}
\hline 
Graphe & Nb noeuds & Temps\tabularnewline
\hline 
\hline 
g50 & 5593 & 1.343795 secondes\tabularnewline
\hline 
g100 & 21353 & 30.570352 secondes\tabularnewline
\hline 
\end{tabular}
\par\end{centering}



}
\par\end{centering}

\caption{Comparaison alpha: Nombre de noeuds visit�s avec la relaxation lagrangienne
(nombre interactions = 10 borne = 5000 gama=2). }
\end{table}


\begin{table}
\begin{centering}
\subfloat[nombre interactions = 20]{\begin{centering}
\begin{tabular}{|c|c|c|}
\hline 
Graphe & Nbnoeuds & Temps\tabularnewline
\hline 
\hline 
g50 & 3289 & 1.132827 secondes\tabularnewline
\hline 
g100 & 16268 & 29.078579 secondes\tabularnewline
\hline 
\end{tabular}
\par\end{centering}

}\subfloat[nombre interactions = 5]{\begin{centering}
\begin{tabular}{|c|c|c|}
\hline 
Graphe & Nb noeuds & Temps\tabularnewline
\hline 
\hline 
g50 & 8631 & 1.465777 secondes\tabularnewline
\hline 
g100 & 30065 & 32.058126 secondes\tabularnewline
\hline 
\end{tabular}
\par\end{centering}

}
\par\end{centering}

\caption{Comparaison nombre iteraction: Nombre de noeuds visit�s avec la relaxation
lagrangienne (Alpha=1.5 borne = 5000 gama=2). }
\end{table}


\begin{table}
\begin{centering}
\begin{tabular}{|c|c|}
\hline 
Alpha & Temps\tabularnewline
\hline 
\hline 
0.1 & 5.158214 secondes\tabularnewline
\hline 
0.3 & 3.011542 secondes\tabularnewline
\hline 
0.5 & 1.501771 secondes\tabularnewline
\hline 
0.75 & 1.334797 secondes\tabularnewline
\hline 
1 & 1.197817 secondes\tabularnewline
\hline 
1.3 & 1.146825 secondes\tabularnewline
\hline 
1.4 & 1.134827 secondes\tabularnewline
\hline 
1.5 & 1.038842 secondes\tabularnewline
\hline 
1.6 & 1.123829 secondes\tabularnewline
\hline 
1.8 & 1.087834 secondes\tabularnewline
\hline 
2 & 1.334797 secondes\tabularnewline
\hline 
2.5 & 1.309800 secondes\tabularnewline
\hline 
3 & 1.485774 secondes\tabularnewline
\hline 
\end{tabular}
\par\end{centering}

\caption{Comparaison alpha: Graphe g50 :Nombre de noeuds visit�s avec la relaxation
lagrangienne (nombre interactions = 10 borne = 5000 gama=2). }
\end{table}


\begin{table}
\begin{centering}
\subfloat[borne = 200000]{\begin{centering}
\begin{tabular}{|c|c|c|}
\hline 
Graphe & Nb noeuds & Temps\tabularnewline
\hline 
\hline 
g50 & 5466 & 1.254809 secondes \tabularnewline
\hline 
g100 & 21292  & 28.038737 secondes \tabularnewline
\hline 
\end{tabular}
\par\end{centering}

}\subfloat[borne = 500]{\begin{centering}
\begin{tabular}{|c|c|c|}
\hline 
Graphe & Nb noeuds & Temps\tabularnewline
\hline 
\hline 
g50 & 4879 & 1.077836 secondes\tabularnewline
\hline 
g100 & 22777 & 25.691094 secondes\tabularnewline
\hline 
\end{tabular}
\par\end{centering}



}
\par\end{centering}

\caption{Comparaison borne: Nombre de noeuds visit�s avec la relaxation lagrangienne
(Alpha=1.5 nombre interactions = 10 gama=2). }
\end{table}


\begin{table}
\begin{centering}
\begin{tabular}{|c|c|}
\hline 
Nb iteration & Temps\tabularnewline
\hline 
\hline 
2 & 3.753429 secondes\tabularnewline
\hline 
3 & 2.238659 secondes\tabularnewline
\hline 
4 & 1.715739 secondes \tabularnewline
\hline 
5 & 1.467776 secondes\tabularnewline
\hline 
10 & 1.032842 secondes\tabularnewline
\hline 
12 & 1.087834 secondes\tabularnewline
\hline 
15 & 0.981850 secondes\tabularnewline
\hline 
18 & 1.432782 secondes \tabularnewline
\hline 
20 & 1.126828 secondes\tabularnewline
\hline 
25 & 1.495772 secondes\tabularnewline
\hline 
50 & 1.861716 secondes\tabularnewline
\hline 
100 & 1.781729 secondes\tabularnewline
\hline 
\end{tabular}
\par\end{centering}

\caption{Comparaison nb iteration: Graphe g50 :Nombre de noeuds visit�s avec
la relaxation lagrangienne (nombre interactions = 10 borne = 5000
gama=2). }
\end{table}


\begin{table}
\begin{centering}
\begin{tabular}{|c|c|}
\hline 
Borne & Temps\tabularnewline
\hline 
\hline 
300 & 0.820875 secondes\tabularnewline
\hline 
500 & 0.720891 secondes\tabularnewline
\hline 
800 & 1.254809 secondes\tabularnewline
\hline 
1000 & 1.071837 secondes\tabularnewline
\hline 
2000 & 1.188819 secondes\tabularnewline
\hline 
3000 & 1.153824 secondes\tabularnewline
\hline 
5000 & 0.990849 secondes\tabularnewline
\hline 
10000 & 0.953854 secondes\tabularnewline
\hline 
20000 & 1.417784 secondes\tabularnewline
\hline 
100000 & 1.083835 secondes\tabularnewline
\hline 
\end{tabular}
\par\end{centering}

\caption{Comparaison nb iteration: Graphe g50 :Nombre de noeuds visit�s avec
la relaxation lagrangienne (alpha = 1.5 nombre interactions = 15 gama=2). }
\end{table}


\begin{table}
\begin{centering}
\begin{tabular}{|c|c|}
\hline 
Gama & Temps\tabularnewline
\hline 
\hline 
1 & Temps non raisonnables\tabularnewline
\hline 
1.5 & Temps non raisonnables\tabularnewline
\hline 
2 & 0.707892 secondes\tabularnewline
\hline 
2.5 & 2.824570 secondes\tabularnewline
\hline 
3 & 0.706892 secondes\tabularnewline
\hline 
4 & 10.137458 secondes\tabularnewline
\hline 
\end{tabular}
\par\end{centering}

\caption{Comparaison nb iteration: Graphe g50 :Nombre de noeuds visit�s avec
la relaxation lagrangienne (alpha = 1.5 nombre interactions = 15 gama=2). }
\end{table}

\end{document}
