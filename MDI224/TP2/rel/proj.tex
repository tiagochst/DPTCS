% -%-%-%-%-%-%-%-%-%-%-%-%-%-%-%-%-%-%-%-%-%-%-%-%-%
% MDI224 % 
% Data:12/12/2011                                 %
% Paris,France                                    % 
% Groupe:                                         %
% - Tiago Chedraoui Silva                   % 
% - Anthony CLERBOUT
% -%-%-%-%-%-%-%-%-%-%-%-%-%-%-%-%-%-%-%-%-%-%-%-%-%

\documentclass[a4paper,11pt]{article}

\usepackage[francais,listings,algo]{tcs}

% Cover %
\def \ttprofname{Roland BADEAU} % teachers name
\def \ttabrv{MDI224} % abbreviation of names class
\def \ttabrvxt{} % period
\def \mytitle{Interpolation par splines cubiques} % Big title
\def \mysubtitle{ Travaux Pratique 2 - Deuxième semestre de 2011} % subtitle
\def \ttauthi{Anthony CLERBOUT} % author's name
\def \ttxti{Casier: 234} % Extra text right side of name
\def \ttauthii{Tiago CHEDRAUOI SILVA} % author's name
\def \ttxtii{Casier: 214 } % Extra text right side of name
\def \ttdate{Décembre 15, 2011} % date

\begin{document}
\titleTMB 
\newpage
\tableofcontents
\listoffigures
\newpage

\section{Résolution du système linéaire}

\subsection{Méthode du gradient à pas constant}

\subsubsection{Implémentation}

Pour voir la méthode du gradient à pas constant, on a fait le code suivant:

\begin{multicols}{2}
  \lstinputlisting[title=\textbf{Méthode du gradient à pas constant}]{../mygradient.m}
\end{multicols}


\subsection{Méthode du gradient à pas optimal}

\subsubsection{Implémentation}
Pour voir la méthode du gradient à pas optimal, on a fait le code suivant:

\begin{multicols}{2}
  \lstinputlisting[title=\textbf{Méthode du gradient à pas optimal}]{../gradient_optimal.m}
\end{multicols}


\subsection{Méthode du gradient conjugué}
\subsubsection{Implémentation}

Pour voir la méthode du gradient conjuge, on a fait le code suivant:
 
\begin{multicols}{2}
  \lstinputlisting[title=\textbf{Méthode du gradient conjugué}]{../gradient_conjugue.m}
\end{multicols}

\end{document}

