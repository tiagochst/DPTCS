%-%-%-%-%-%-%-%-%-%-%-%-%-%-%-%-%-%-%-%-%-%-%-%-%-%
% MDI200: Probabilités                            %  
% Devoir 1                                        %
% Data:05/10/2011                                 %
% Paris,France                                    % 
% Groupe:                                         %
%       - Tiago Chedraoui Silva                   % 
%-%-%-%-%-%-%-%-%-%-%-%-%-%-%-%-%-%-%-%-%-%-%-%-%-%

\documentclass[a4paper]{article}

%%% fontes %%%
\usepackage[french]{babel}
\usepackage[T1]{fontenc}
\usepackage[utf8]{inputenc}   % acentuação
\usepackage{ae,aecompl,aeguill}  % pdfs plus beaux =)
\usepackage{dsfont}

%%% autres %%%
\usepackage{multirow}
\usepackage{textcomp}
\usepackage{color}       
\usepackage{indentfirst}     
\usepackage{multicol}   
\usepackage[linkbordercolor={1 1 1},urlcolor=black,colorlinks=true]{hyperref} % links
\usepackage{subfig}
\usepackage[letterpaper]{geometry}
\geometry{verbose,lmargin=3cm,rmargin=3cm}
\usepackage{amsfonts}

%\setcounter{tocdepth}{0}

\renewcommand{\thefigure}{\arabic{figure}}

\date{Octobre 17, 2011}
% Capa estilizada %
\newcommand*{\titleTMB}{\begingroup \centering \settowidth{\unitlength}{\LARGE EE531} {\large\scshape MDI200 - Cycle d'harmonisation }\\[0.2\baselineskip] \rule{11.0cm}{1.6pt}\vspace*{-\baselineskip}\vspace*{2pt} \rule{11.0cm}{0.4pt}\\[\baselineskip] {\LARGE Devoir 01}\\\vspace*{\baselineskip}  {\itshape Introduction aux Probabilités - Deuxième Semestre de 2011}\\ \rule{11.0cm}{0.4pt}\vspace*{-\baselineskip}\vspace{3.2pt} \rule{11.0cm}{1.6pt}\\[\baselineskip] {\large\scshape Professeur: Michel Grojnowski}\par \vfill {\normalsize   \scshape 
    \begin{center} 
      \begin{tabular}{  l  l  p{5cm} } 
        Tiago Chedraoui Silva  & Cassier: 214\\
      \end{tabular} \end{center}
    \itshape 17 octobre de 2011    }\\[\baselineskip] \vspace{3.2pt} \endgroup}


\begin{document}
\titleTMB 
\newpage

\section{Exercice 1}

\subsection{}
Si $X$ et $Y$ sont deux variables aléatoires indépendantes de lois respectives $G(a_1,\mu)$ et $G(a_2,\mu)$, cela veut dire:

\begin{equation}
f(x)=\mathds{1}_{\mathbb{R}+}(x)\frac{\mu^{a_1}}{\Gamma (a_1)}x^{a_1-1}e^{-\mu x}
\end{equation}

\begin{equation}
f(y)=\mathds{1}_{\mathbb{R}+}(x)\frac{\mu^{a_2}}{\Gamma (a_2)}x^{a_2-1}e^{-\mu x}
\end{equation}

Où:
\begin{equation}
\Gamma \left( a \right) = \int\limits_0^\infty {x^{a - 1} } e^{ - x} dx  
\end{equation}

En utilizant la convolution, on doit trouver la loi de probabilité d'une somme de deux variables indépendantes $Z = X + Y$. 

Donc, soit k(z) une densité on a:

\begin{equation}
f(z)=f(x)f(y)
\end{equation}
\begin{equation}
k(z)= \int\limits_0^\infty\frac{\mu^{a_2}}{\Gamma (a_2)}x^{a_2-1}e^{-\mu x} \frac{\mu^{a_1}}{\Gamma (a_1)}(z-x)^{a_1-1}e^{-\mu (z-x)}dx
\end{equation}

\begin{equation}
k(z)=\frac{\mu^{a_2+a_1}}{\Gamma (a_2)\Gamma (a_1)}\int\limits_0^\infty{x^{a_2-1}e^{-\mu x} }(z-x)^{a_1-1}e^{-\mu (z-x)}dx
\end{equation}


\begin{equation}
k(z)=\frac{\mu^{a_2+a_1}e^{-\mu z}}{\Gamma (a_2)\Gamma (a_1)}\int\limits_0^\infty{x^{a_2-1}}(z-x)^{a_1-1}dx
\end{equation}

Si $x = tz$ (Donc, $dx = zdt$)

\begin{equation}
k(z)=\frac{\mu^{a_2+a_1}e^{-\mu z}}{\Gamma (a_2)\Gamma (a_1)}\int\limits_0^1{t^{a_2-1}z^{a_2-1}}(z^{a_1-1}(1-t)^{a_1-1})zdt
\end{equation}

\begin{equation}
k(z)=\frac{\mu^{a_2+a_1}e^{-\mu z}z^{a_2+a_1-1}}{\Gamma (a_2)\Gamma (a_1)}\int\limits_0^1{t^{a_2-1}}(1-t)^{a_1-1}dt
\end{equation}

En connaissant la fonction Beta:

\begin{equation}
\mathrm{\beta}(x,y) = \int_0^1t^{x-1}(1-t)^{y-1}\,dt
\end{equation}

Et la propriéte:

\begin{equation}
\beta(x,y)=\frac{\Gamma(x)\,\Gamma(y)}{\Gamma(x+y)}
\end{equation}


On a:
\begin{equation}
k(z)=\frac{\mu^{a_2+a_1}e^{-\mu z}z^{a_2+a_1-1}}{\Gamma (a_2)\Gamma (a_1)}\frac{\Gamma(a_1)\,\Gamma(a_2)}{\Gamma(a_1+a_2)}
\end{equation}

Finalment:

\begin{equation}
k(z)=\frac{\mu^{a_2+a_1}e^{-\mu z}z^{a_2+a_1-1}}{\Gamma(a_1+a_2)}
\end{equation}

Ainsi:
\begin{equation}
\gamma_{x}+\gamma_{y}=\gamma_{x+y}
\end{equation}

Par la démonstration au dessus, le résultat ne serais pas sembable si les segondes paramètres ne sont pas le mêmes, parce que 
$\mu_{1}^r \mu_{2}^s = \mu^{s+r}$ si et seulement si $\mu_1=\mu_2=\mu$.

\subsection{}

Soit X une v.a de loi $N(0,\sigma^2)$:
\begin{equation}
f(x)=\frac{1}{\sigma\sqrt{2\pi}}e^{(-\frac{1}{2}(\frac{x-\mu}{\sigma})^2)}
\end{equation}
\begin{equation}
f(x)=\frac{1}{\sigma\sqrt{2\pi}}e^{(-\frac{1}{2}(\frac{x}{\sigma})^2)}
\end{equation}

Donc, si $Z=X*X$ ($dZ=2XdX$)

$\frac{U^2}{2}$ suit une loi de $\gamma_{\frac{1}{2}}$ 

\subsection{}
\subsection{}
\subsection{}

%\begin{enumerate}
%\item oi
%\end{enumerate}

\section{Exercice 2 }
\subsection{}
\subsection{}
\subsection{}
\subsection{}
\subsection{}


\end{document}