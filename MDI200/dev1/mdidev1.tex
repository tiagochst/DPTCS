%-%-%-%-%-%-%-%-%-%-%-%-%-%-%-%-%-%-%-%-%-%-%-%-%-%
% MDI200: Probabilités                            %  
% Devoir 1                                        %
% Data:05/10/2011                                 %
% Paris,France                                    % 
% Groupe:                                         %
%       - Tiago Chedraoui Silva                   % 
%-%-%-%-%-%-%-%-%-%-%-%-%-%-%-%-%-%-%-%-%-%-%-%-%-%

\documentclass[a4paper]{article}

%%% fontes %%%
\usepackage[french]{babel}
\usepackage[T1]{fontenc}
\usepackage[utf8]{inputenc}   % acentuação
\usepackage{ae,aecompl,aeguill}  % pdfs plus beaux =)
\usepackage{dsfont}

%%% autres %%%
\usepackage{multirow}
\usepackage{textcomp}
\usepackage{color}       
\usepackage{indentfirst}     
\usepackage{multicol}   
\usepackage[linkbordercolor={1 1 1},urlcolor=black,colorlinks=true]{hyperref} % links
\usepackage{subfig}
\usepackage[letterpaper]{geometry}
\geometry{verbose,lmargin=3cm,rmargin=3cm}

\usepackage{amsfonts}
\usepackage{amsmath} 

%\setcounter{tocdepth}{0}

\renewcommand{\thefigure}{\arabic{figure}}

\date{Octobre 17, 2011}
% Capa estilizada %
\newcommand*{\titleTMB}{\begingroup \centering \settowidth{\unitlength}{\LARGE EE531} {\large\scshape MDI200 - Cycle d'harmonisation }\\[0.2\baselineskip] \rule{11.0cm}{1.6pt}\vspace*{-\baselineskip}\vspace*{2pt} \rule{11.0cm}{0.4pt}\\[\baselineskip] {\LARGE Devoir 01}\\\vspace*{\baselineskip}  {\itshape Introduction aux Probabilités - Deuxième Semestre de 2011}\\ \rule{11.0cm}{0.4pt}\vspace*{-\baselineskip}\vspace{3.2pt} \rule{11.0cm}{1.6pt}\\[\baselineskip] {\large\scshape Professeur: Michel Grojnowski}\par \vfill {\normalsize   \scshape 
    \begin{center} 
      \begin{tabular}{  l  l  p{5cm} } 
        Tiago Chedraoui Silva  & Cassier: 214\\
      \end{tabular} \end{center}
    \itshape 17 octobre de 2011    }\\[\baselineskip] \vspace{3.2pt} \endgroup}


\begin{document}
\titleTMB 
\newpage

\section{Exercice 1}

\subsection{}
Si $X$ et $Y$ sont deux variables aléatoires indépendantes de lois respectives $G(a_1,\mu)$ et $G(a_2,\mu)$, cela veut dire:

\begin{equation}
f(x)=\mathds{1}_{\mathbb{R}+}(x)\frac{\mu^{a_1}}{\Gamma (a_1)}x^{a_1-1}e^{-\mu x}
\end{equation}

\begin{equation}
f(y)=\mathds{1}_{\mathbb{R}+}(x)\frac{\mu^{a_2}}{\Gamma (a_2)}x^{a_2-1}e^{-\mu x}
\end{equation}

Où:
\begin{equation}
\Gamma \left( a \right) = \int\limits_0^\infty {x^{a - 1} } e^{ - x} dx  
\end{equation}

En utilizant la convolution, on doit trouver la loi de probabilité d'une somme de deux variables indépendantes $Z = X + Y$. 

Donc, soit K(z) une loi de probabilité on a:

\begin{equation}
f(z)=f(x)f(y)
\end{equation}

\begin{eqnarray*}
K(z) &=& \int\limits_0^\infty\frac{\mu^{a_2}}{\Gamma (a_2)}x^{a_2-1}e^{-\mu x} \frac{\mu^{a_1}}{\Gamma (a_1)}(z-x)^{a_1-1}e^{-\mu (z-x)}dx\\
&=&\frac{\mu^{a_2+a_1}}{\Gamma (a_2)\Gamma (a_1)}\int\limits_0^\infty{x^{a_2-1}e^{-\mu x} }(z-x)^{a_1-1}e^{-\mu (z-x)}dx\\
&=&\frac{\mu^{a_2+a_1}e^{-\mu z}}{\Gamma (a_2)\Gamma (a_1)}\int\limits_0^\infty{x^{a_2-1}}(z-x)^{a_1-1}dx
\end{eqnarray*}

Si $x = tz$ (Donc, $dx = zdt$)

\begin{eqnarray*}
K(z)&=&\frac{\mu^{a_2+a_1}e^{-\mu z}}{\Gamma (a_2)\Gamma (a_1)}\int\limits_0^1{t^{a_2-1}z^{a_2-1}}(z^{a_1-1}(1-t)^{a_1-1})zdt\\
&=&\frac{\mu^{a_2+a_1}e^{-\mu z}z^{a_2+a_1-1}}{\Gamma (a_2)\Gamma (a_1)}\int\limits_0^1{t^{a_2-1}}(1-t)^{a_1-1}dt
\end{eqnarray*}

En connaissant la fonction Beta:

\begin{equation}
\mathbb{\beta}(x,y) = \int_0^1t^{x-1}(1-t)^{y-1}\,dt
\end{equation}

Et la propriéte:

\begin{equation}
\beta(x,y)=\frac{\Gamma(x)\,\Gamma(y)}{\Gamma(x+y)}
\end{equation}

On a:
\begin{equation*}
K(z)=\frac{\mu^{a_2+a_1}e^{-\mu z}z^{a_2+a_1-1}}{\Gamma (a_2)\Gamma (a_1)}\frac{\Gamma(a_1)\,\Gamma(a_2)}{\Gamma(a_1+a_2)}
\end{equation*}

Finalment:

\begin{equation}
K(z)=\frac{\mu^{a_2+a_1}e^{-\mu z}z^{a_2+a_1-1}}{\Gamma(a_1+a_2)}
\end{equation}

Ainsi:
\begin{equation*}
\gamma_{x}+\gamma_{y}=\gamma_{x+y}
\end{equation*}

Par la démonstration au dessus, le résultat ne serais pas sembable si les segondes paramètres ne sont pas le mêmes, parce que 
$\mu_{1}^r \mu_{2}^s = \mu^{s+r}$ si et seulement si $\mu_1=\mu_2=\mu$.

On peut aussi utilizer la fonction caracteristique:
\begin{equation}
\phi_{x_1+x_2}=\phi_{x_1}+\phi_{x_2}=\frac{1}{(1-\frac{it}{\mu})^{a_1+a_2}}
\end{equation}

Donc, on a une loi $G(a_1+a_2,\mu)$


%% Questao 2 parte 1

\subsection{}

Soit X une v.a de loi $N(0,\sigma^2)$, la v.a. $X^2$ suit une loi $G(\frac{1}{2},\frac{1}{2\sigma^2})$:

\begin{eqnarray*}
E[g(X^2)]&=&\frac{1}{\sigma\sqrt{2\pi}}\int\limits_{-\infty}^\infty g(x^2)e^{-\frac{x^2}{2\sigma²}}dx\\
&=&\frac{2}{\sigma\sqrt{2\pi}}\int\limits_{0}^\infty g(x^2)e^{-\frac{x^2}{2\sigma²}}dx
\end{eqnarray*}

Chagement de variable: $t=x^2$ ($dt=2xdx$)

\begin{equation}
E[g(t)]=\frac{2}{\sigma\sqrt{2\pi}}\int\limits_{0}^\infty g(t)t^{-\frac{1}{2}}e^{-\frac{t}{2\sigma²}}dt
\end{equation}

Donc, la v.a $X²$ a une loi $G(\frac{1}{2},\frac{1}{2\sigma^2})$.

% Questão 3 parte 1
\subsection{}
Soient $X_1,...,X_n$  des variables  aléatoires indépendantes suivant  toutes la
même loi  $N(0, \sigma^2)$.  La loi,  et la densité  de la  v.a $Y=X_1^2  +... +
X_n^2$ suite une loi $G(\frac{n}{2},\frac{1}{2})$, que s'appelle aussi, loi du khi-carré.

Ainsi, la densité est:

\begin{equation}
f(t)=\frac{1}{2^{\frac{n}{2}}\Gamma(\frac{n}{2})}t^{\frac{n}{2}-1}e^{-\frac{t}{2}}
\end{equation}

Et, la loi est:
\begin{equation}
F(t)=\frac{\gamma(n/2,1/2)}{\Gamma(n/2)}=G(\frac{n}{2},\frac{1}{2}) 
\end{equation}


%%% questão 4 parte 1

\subsection{}
\subsubsection{}
Soit T une  v.a. à valeurs dans $\mathbb{R}_+$  vérifiant la condition d'absence
d'usure (non-vieillissement), on  a pour $t=0$, ça veut dire qui  le début est 0
et donc, il n'existe pas de difference entre las probabilités de deux temps (non-vieillissement):
\begin{equation}
P(T\geq s+t/ T \geq s)=P(T\geq t /T\geq 0)=P(T\geq t )
\end{equation}
Parce  que $T\geq  t=0$  et $P[T\geq  0]=1$. Et  pout  $t\geq 0$  et sachant  la
probabilité condicionel:

\begin{equation}
P(T\geq s+t/ T\geq s)=\frac{P[(T\geq s+t )\cap (T \geq s)]}{P(T\geq s )}=\frac{P(T\geq s+t )}{P(T\geq s )}
\end{equation}

Avec les equations antérieurs, on a: 

\begin{equation}
P(T\geq s )=\frac{P(T\geq s+t )}{P(T\geq t )}
\end{equation}

Si $G(t)=P(T\geq t )$ donc: 

\begin{equation}
G(s)=\frac{G(s+t)}{G(t)}, (G(0)= 1)
\end{equation}

\subsubsection{}
Soi G(s) la  loi de probabilité, pour  avoir la dénsité de cette  loi, G(s) doit
être derivable. Ainsi, la unique function (et sa derivé) dont le produit entre deux
functions (dérivées) est le même que la funciotion (derivée) de la somme est une
exponencial:

\begin{equation}
G(t)=e^{\lambda t}
\end{equation}
Donc, la derivé (fonction de dénsité):

\begin{equation}
f(t)=\lambda e^{\lambda t}
\end{equation}

Comme f(t)  doi être plus petite  que 1, $f(t)\leq  1$, donc on sait  que, comme
$t\geq 0$, $\lambda \leq 0$.

Pour  considerer que  $t  \in \mathbb{R}$,  on  sache que  tout  réel peut  être
considéré  comme  une limite  d'une  suite de  rationnels,  et  que la  fonction
exponencial est continue sur $\mathbb{R}$,  donc on peut élagir la notation pour
les réels ($\mathbb{R}$).

\section{Exercice 2 }


\subsection{}
En sachant que la loi de probabilité de poisson est donné par:

\begin{equation}
P(X=k )=\frac{e^{-\lambda }(\lambda )^k}{k!},k \in \mathbb{N}
\end{equation}

Étant un processus ponctuel de poisson tel que, pour tout instant t, la v.a. $N_{0,t}$,
suit une loi de Poisson de paramètre $\lambda t$, $\lambda >0$, on a:

\begin{equation*}
P(N_{0,t}=k)=\frac{e^{-\lambda (t-0)}(\lambda (t-0))^k}{k!}=\frac{e^{-\lambda t}(\lambda t)^k}{k!} , k \in \mathbb{N}
\end{equation*}

Ainsi,
\begin{equation*}
P(N_{0,t}-N_{0,s}=N_{s,t}=k)=\frac{e^{-\lambda (t-s)}(\lambda (t-s))^k}{k!} , k \in \mathbb{N}
\end{equation*}

On voit que $N_{0,t}=N_{0,s}+N_{s,t}$, veut dire que le nombre d'arrivés entre 0
et t, est  la somme entre les temps 0 et  s; et s et t. Donc,  le dombre dans un
periode s et t, irais être le nombre entre 0 et t moins le nombre entre 0 et s.

En utilisant la fonction caracteristique on a:
\begin{equation}
\phi_{x_1}=E[e^{itx}]=e^{\lambda t_1(e^{it}-1)}
\end{equation}

Donc,    comme     $N_{0,s}$    et    $N_{s,t}$     sont    indépendentes,    soit
$Y=N_{0,t}$,$X_1=N_{0,s}$ et $X_2=N_{s,t}$. On a:
\begin{equation*}
\phi_{Y=X_2+X_1}=\phi_{X_2}\phi_{X_1}=e^{\lambda (t+s)(e^{in}-1)}
\end{equation*}

Étant donné cela, on voit que la fonction caracteristique corresponde a une 
fonction de poisson avec les parametres t+s.

Ainsi, on peut vérifier que :

\begin{equation}
P(N_{0,t}-N_{0,s}=N_{s,t}=k)=\frac{e^{-\lambda (t-s)}(\lambda (t-s))^k}{k!} , k \in N
\end{equation}


%% Parte dois questão 1
\subsection{}
On sache que $T_1$ est la première arrivée, donc $P[T_1\geq t]=P[N_{0,t}=0]$, parce
que jusqu'au moment $t$ on doit avoir  aucune arivée pour que au moment un petit
peu plus grand que t, on a la première arrivée.

Ainsi:

\begin{equation*}
P(N_{0,t}=0)=\frac{e^{-\lambda (t-0)}(\lambda (t-0))^0}{0!}=e^{-\lambda t} 
\end{equation*}


On sait que $t_n= X_1 + X_2+ ...+ x_n$, cela veut dire que le temps $t_n$ est le
temps necessaire pour que on a n arrivées. Comme un arrive a une loi exponentiel
(il respect la loi de poisson), $t_n$ serait donne par une somme de v.a. dont la
loi de  densité est exponenciel. Ainsi, la  densité et la loi  d'un processus de
poisson est donné respectivement par:

\begin{equation}
f_n(x)=\frac{\lambda^{n}}{(n-1)!}x^{n-1}e^{-\lambda x}
\end{equation}

Preuve:


\begin{equation} 
P[T_n > t] = \sum_{k=0}^{n-1}e^{-\lambda t}\frac{(\lambda t)^k}{k!}
\end{equation}

\begin{equation}\label{eq:e2q2}
P[T_{n+1} >  t] = \sum_{k=0}^{n}e^{-\lambda t}\frac{(\lambda  t)^k}{k!}= P[T_n >
t]+ \frac{e^{-\lambda (t)}(\lambda (t))^n}{n!}
\end{equation}

En utilizant l'équation \ref{eq:e2q2}, on peut dire que :
\begin{equation} 
P[T_n  > t] =  \frac{\lambda^{n}}{(n-1)!}\int\limits_0^\infty x^{n-1}e^{-\lambda
  x}dx
\end{equation}

Si et seulement si l'équation suivante est vérifié:
\begin{equation} 
\frac{\lambda^{n+1}}{(n)!}\int\limits_0^\infty x^{n}e^{-\lambda
  x}dx = \frac{\lambda^{n}}{(n-1)!}\int\limits_0^\infty x^{n-1}e^{-\lambda
  x}dx +  \frac{e^{-\lambda t}(\lambda t)^n}{n!}
\end{equation}

Par integration  par partie,  on a $  x^n =  U$, $nx^{n-1}=dU$ et  $ e^{-\lambda
  x}dx=dV$, $V=\frac{e^{-\lambda x}}{-\lambda}$

Ainsi:
% \begin{equation} 
% \frac{\lambda^{n+1}}{(n)!}\int\limits_0^\infty x^{n}e^{-\lambda
%   x}dx       =\frac{\lambda^{n+1}}{n!}[VU  -       \int       VdU]       =
% \frac{\lambda^{n+1}}{n!}[\frac{x^n    e^{-\lambda     x}}{-\lambda}    -    \int
% \frac{x^{n-1}n e^{-\lambda x}}{-\lambda}dx] = 
% \end{equation}

% \begin{equation} 
% = \frac{\lambda^{n}}{(n-1)!}[\frac{-x^n e^{-\lambda x}}{n} +
% \int_t^\infty{x^{n-1} e^{-\lambda x}}dx]
% \end{equation}


\begin{eqnarray*}
\frac{\lambda^{n+1}}{(n)!}\int\limits_0^\infty x^{n}e^{-\lambda
  x}dx &=& \frac{\lambda^{n+1}}{n!}[VU|_t^\infty - \int_t^\infty VdU] \\
& =&\frac{\lambda^{n+1}}{n!}[\frac{x^n    e^{-\lambda     x}}{-\lambda}    -    \int
\frac{x^{n-1}n e^{-\lambda x}}{-\lambda}dx] \\
& =&  \frac{\lambda^{n}}{(n-1)!}[\frac{-x^n e^{-\lambda x}}{n}|_t^\infty +
\int_t^\infty{x^{n-1} e^{-\lambda x}}dx] \\
& =&  \frac{\lambda^{n}}{(n-1)!}[\frac{-t^n e^{-\lambda t}}{n} +
\int_t^\infty{x^{n-1} e^{-\lambda x}}dx] \\
& =& \frac{(\lambda t)^n e^{-\lambda t}}{n!} +
\frac{\lambda^{n}}{(n-1)!}\int_t^\infty{x^{n-1} e^{-\lambda x}}dx
\end{eqnarray*}

Donc, par récurrence  on a démontrer que la formule  de probabilité est vraiment
aplicable a $T_n$, et de même façon on a vérifié que $T_n$ suit une loi $G(n,\lambda)$.

\begin{equation}
G(n,\lambda)= \int\limits_0^\infty \frac{\lambda^{n}}{\Gamma (n)}x^{n-1}e^{-\lambda x}dx=\frac{\lambda^{n}}{(n-1)!}\int\limits_0^\infty x^{n-1}e^{-\lambda x}dx=P[T_n > t]
\end{equation}


% Parte 3 q2
\subsection{}
Soit $T_1$ et $T_2$ independantes, on a que $\phi_{t+(-s)}=\phi_t*\phi_{-s}$:

\begin{equation}
f_n(x)=\frac{\lambda^{n}}{(n-1)!}x^{n-1}e^{-\lambda x}
\end{equation}

\begin{equation}
\phi_{x_1}=E(e^{tix})=\int\limits_0^\infty f_n(e^{itx})xdx=e^{\lambda t_1(e^{it}-1)}
\end{equation}

\begin{equation}
\phi_{x_1}=E(e^{tix})=\int\limits_0^\infty f_n(e^{itx})xdx=e^{\lambda t_1(e^{it}-1)}
\end{equation}

\begin{equation}
\phi_{T_2-T_1}=\phi_t*\phi_{-s}=e^{\lambda (t_2-t_1)(e^{it}-1)}
\end{equation}



\subsection{}

On sait que:

\begin{equation}
P[T_2 > t] =1 - P[T_2 < t]
\end{equation}

Nous voudrions  chercher la probabilite tel  que $P[T_1\leq s  \cap T_2\leq t]$,
donc si t ou s = 0, on a $s \cap t = 0$ et cela nous donne une probabilité 0.

Pour analiser les autres situations, on a:

\begin{equation*}
P[T_2 > t] = \frac{\lambda ^2}{(2 - 1)!}\int\limits_t^\infty{x^{2-1}}e^{-\lambda x}dx= \lambda ^2 *(-\frac{x}{\lambda} e^{-\lambda x} |_t^\infty)- \lambda^{-2} e^{-\lambda x}|_t^\infty = \lambda te^{-\lambda t} + e^{-\lambda t}
\end{equation*}

Comme  on voudrait  la probabilité  d'intersection  entre s  e t,  si $s<t$,  la
probabilité de que  les évenements sont avant  de n, et aussi avant  t est donné
par, $1 - \lambda se^{-\lambda s} + e^{-\lambda s}$, si $t<s$, on a 
$1 - \lambda te^{-\lambda t} + e^{-\lambda t}$.

%Et on a déduit que 

%\begin{equation}
%P[T_1 > t] = e^{-\lambda }
%\end{equation}

Conclusion: 

Si $s \leq 0$ ou $t \leq 0$, donc $P[{T_1 \leq s} \cap {T_2 \leq t}]=0$. Donc,(F(s,t)=0).

Si $0 \leq s \leq t $, donc $P[{T_1 \leq s} \cap {T_2 \leq t}]=1-P[T_2 > s]$. Donc,(F(s,t)=$1 - \lambda se^{-\lambda s} - e^{-\lambda s}$).

Si $0 \leq t \leq s $, donc $P[{T_1 \leq s} \cap {T_2 \leq t}]=1-P[T_2 > t]$. Donc,(F(s,t)=$1 - \lambda te^{-\lambda t} - e^{-\lambda t}$).


\subsection{}
Avec la fonction de repation, on  peut avoir la dénsité en dérivante la fonction
de repartition. 

\begin{equation}
f_x(t) = \lambda e^{\lambda s}+\lambda ^2se^{-\lambda s}+\lambda e^{-\lambda t} , t>0
\end{equation}

\begin{equation}
f_x(t_1-t_2) = \lambda e^{\lambda t_1-t_2}+\lambda ^2(t_1-t_2)e^{-\lambda t_1-t_2}+\lambda e^{-\lambda t_1-t_2} , t>0
\end{equation}


%\section{Bibliography}

%IEOR 6711: Notes on the Poisson Process


\end{document}