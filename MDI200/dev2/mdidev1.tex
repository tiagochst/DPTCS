% -%-%-%-%-%-%-%-%-%-%-%-%-%-%-%-%-%-%-%-%-%-%-%-%-%
% MDI200: Probabilités                            %  
% Devoir 1                                        %
% Data:05/10/2011                                 %
% Paris,France                                    % 
% Groupe:                                         %
% - Tiago Chedraoui Silva                   % 
% -%-%-%-%-%-%-%-%-%-%-%-%-%-%-%-%-%-%-%-%-%-%-%-%-%

\documentclass[a4paper]{article}

%%% fontes %%%
\usepackage[french]{babel}
\usepackage[T1]{fontenc}
\usepackage[utf8]{inputenc}   % acentuação
\usepackage{ae,aecompl,aeguill}  % pdfs plus beaux =)
\usepackage{dsfont}

%%% autres %%%
\usepackage{multirow}
\usepackage{textcomp}
\usepackage{color}       
\usepackage{indentfirst}     
\usepackage{multicol}   
\usepackage[linkbordercolor={1 1 1},urlcolor=black,colorlinks=true]{hyperref} % links
\usepackage{subfig}
\usepackage[letterpaper]{geometry}
\geometry{verbose,lmargin=3cm,rmargin=3cm}

\usepackage{amsfonts}
\usepackage{amsmath} 

\usepackage{pgfplots}

\usepackage{tikz}
\usetikzlibrary{trees}

% Set the overall layout of the tree
\tikzstyle{level 1}=[level distance=3.5cm, sibling distance=3.5cm]
\tikzstyle{level 2}=[level distance=3.5cm, sibling distance=2cm]

% Define styles for bags and leafs
\tikzstyle{bag} = [text width=4em, text centered]
\tikzstyle{end} = [circle, minimum width=3pt,fill, inner sep=0pt]

% Define a the counter cnt. Used to identify files generated for use
% with Gnuplot.
\newcounter{cnt}
\setcounter{cnt}{0}


\newcommand{\distpic}[3]{
    % First draw the upper distribution.
    % Shade the critical region:
    \fill[red!30] (0.658,0)  -- plot[id=f3,domain=0.658:3,samples=50]
        function {exp(-x*x*0.5/0.16)} -- (3,0) -- cycle;

    % Draw the normal distribution curve
    \draw[blue!50!black,smooth,thick] plot[id=f1,domain=-2:3,samples=50]
    function {};
    % Draw the x-axis
    \draw[->,black] (-2.2,0) -- (3.2,0);
    % Put some ticks and tick labels in:
    \foreach \x in {-2,-1,0,1,2,3}
    \draw (\x,0) -- (\x,-0.1) node[below] {$\x$};
    % Put in a label for the critical region boundary:
    \draw[red!50!black,thick] (0.658,0) node[below,yshift=-0.5cm] {0.658}
    -- (0.658,0.85);

    % Put in labels for accepting or rejecting the null hypothesis with
    % the corresponding regions:
    \draw[red!50!black,thick,->] (0.688,0.7) -- (1.3,0.7)
        node[anchor=south] {Reject  $H_0$};
    \draw[red!50!black,thick,->] (0.628,0.7) -- (-1,0.7)
        node[anchor=south]{\parbox{1.5cm}{\raggedright Fail to reject $H_0$}};

    % Add a label to the upper picture, when the null is true
    \draw (-3,1) node[above,draw,fill=green!30] {$H_0$ is true:};

    % Label the critical region with an alpha level:
    \draw[<-,thick] (0.75,0.05) -- (1.6,0.2) node[right,yshift=0.3cm]
    {\begin{tabular}{l} $\alpha=0.05$ \\ (Type I error rate) \end{tabular}};


    % Add a label showing the effect size between the two plots:
    \draw[very thin] (0,-1) -- (0,-0.5);
    \draw[<->,thick] (0,-1) node[left] {Effect size:  #1} -- (#1,-1);
    \draw[thick] (0,-.9) -- (0,-1.1);

    \draw[very thin] (#1,-1) -- (#1,-1.7);
    \draw[thick] (#1,-.9) -- (#1,-1.1);

    % Now draw the lower distribution showing the effect size:
    \begin{scope}[yshift=-3cm]
    % Shade the "reject H0" region red
    \fill[red!30] (0.658,0)  -- plot[id=f3\thecnt,domain=0.658:3,samples=50]
        function {exp(-(x-#1)*(x-#1)*0.5/0.16)} --
        (3,0) -- cycle;
        % Shade the "accept H0" region blue
    \fill[blue!30] (-2,0) -- plot[id=f4\thecnt,domain=-2:0.658,samples=50]
        function {exp(-(x-#1)*(x-#1)*0.5/0.16)} --
        (0.658,0) -- cycle;

        % Draw the shifted normal distribution:
    \draw[blue!50!black,smooth,thick] plot[id=f1\thecnt,domain=-2:3,samples=50]
            function {exp(-(x-#1)*(x-#1)*0.5/0.16)};

        % Draw the x-axis and put in some ticks and tick labels
    \draw[->,black] (-2.2,0) -- (3.2,0);
    \foreach \x in {-2,-1,0,1,2,3}
            \draw (\x,0) -- (\x,-0.1) node[below] {$\x$};

        % Draw and label the critical region boundary
    \draw[red!50!black,very thick] (0.658,0) node[below,yshift=-0.5cm] {0.658}
        -- (0.658,1.0);
    \draw[red!50!black,very thick,->] (0.688,0.7) -- (2.7,0.7)
        node[anchor=south west] {Reject  $H_0$};
    \draw[red!50!black,very thick,->] (0.628,0.7) -- (-0.5,0.7)
        node[anchor=south]{\parbox{1.5cm}{\raggedright Fail to reject $H_0$}};

    % Add a label to the lower picture, when the alternative hypothesis is true:
    \draw (-3,1) node[above,draw,fill=green!30] {$H_a$ is true:};

        % Add labels showing the statistical power and the Type II error rate:
    \draw[<-,thick] (1.5,0.1) -- (3,0.2) node[anchor=south west]
        {Power = \large #2};
    \draw[<-,thick] (0.4,0.1) -- (-1,0.2) node[left,yshift=0.3cm]
        {\begin{tabular}{l}
        $\beta$ = {\large #3} \\ (Type II error rate) \end{tabular}};
    \end{scope}
}

% \setcounter{tocdepth}{0}

\renewcommand{\thefigure}{\arabic{figure}}

\date{Octobre 17, 2011}
% Capa estilizada %
\newcommand*{\titleTMB}{\begingroup \centering \settowidth{\unitlength}{\LARGE EE531} {\large\scshape MDI200 - Cycle d'harmonisation }\\[0.2\baselineskip] \rule{11.0cm}{1.6pt}\vspace*{-\baselineskip}\vspace*{2pt} \rule{11.0cm}{0.4pt}\\[\baselineskip] {\LARGE Devoir 02}\\\vspace*{\baselineskip}  {\itshape Introduction aux Probabilités - Deuxième Semestre de 2011}\\ \rule{11.0cm}{0.4pt}\vspace*{-\baselineskip}\vspace{3.2pt} \rule{11.0cm}{1.6pt}\\[\baselineskip] {\large\scshape Professeur: Michel Grojnowski}\par \vfill {\normalsize   \scshape 
    \begin{center} 
      \begin{tabular}{  l  l  p{5cm} } 
        Tiago Chedraoui Silva  & Cassier: 214\\
      \end{tabular} \end{center}
    \itshape 17 octobre de 2011    }\\[\baselineskip] \vspace{3.2pt} \endgroup}


\begin{document}
\titleTMB 
\newpage

\section{Exercice 1}
\subsection{}

En sachant que A est au service au départ, la probabilité qu'il gagne est, ou il  gagne le  premier  point (p),  ou il  perde  le premier  (p-1),il gagne  le
deuxième (q) e il gagne le  troisième(p) et ainsi de suite. Le dessin ci-dessous
représente les états du jeu que sont dans un cycle.

% The sloped option gives rotated edge labels. Personally
% I find sloped labels a bit difficult to read. Remove the sloped options
% to get horizontal labels. 
\begin{tikzpicture}[grow=right, sloped]
  \node[bag] {Service A}
  child {
    node[bag] {Service B} 
    child {
      node[bag,label=
      {Service A}] 
      {}
      child {
        node[bag] {Service B} 
        child {
          node[end,label=
          {Service A}] {}
          edge from parent
          node[above] {q}
          node[below]  {}
        }
        edge from parent
        node[above] {1-p}
        node[below]  {}
      }
      child {
        node[bag] {Point A} 
        edge from parent         
        node[above] {p}
        node[below]  {}
      }
      edge from parent
      node[above] {q}
      node[below]  {}
    }
    edge from parent 
    node[above] {1-p}
    node[below]  {}
  }
  child {
    node[bag] {Point A} 
    edge from parent         
    node[above] {p}
    node[below]  {}
  };
\end{tikzpicture}


Le tableau ci-dessous
représente le probabilité de un point.

\begin{table}[h!]
  \begin{centering}
    \begin{tabular}{|c|c|c|}
      \hline 
      k & nombre services & Probabilité que A gagne le point après n services\tabularnewline
      \hline 
      \hline 
      0 & 1 & p\tabularnewline
      \hline 
      1 & 3 & (1-p)qp\tabularnewline
      \hline 
      2 & 5 & (1-p)\texttwosuperior{}q\texttwosuperior{}p\tabularnewline
      \hline 
      3 & 7 & (1-p)\textthreesuperior{}q\textthreesuperior{}p\tabularnewline
      \hline 
    \end{tabular}
    \par\end{centering}
  \caption{Probabilite 1 point}
\end{table}

En utilisant le tableau, on peut savoir la $P[n=1]$:

\begin{eqnarray*}
  P(n=1)  &  =  &  \sum_{k=0}(1-p)^kq^kp= \frac{p}{1-(1-p)q}  ;\quad  \text{(suite
    géométrique avec (1-p)q<1) }
\end{eqnarray*}

Pour que A gagne n points, il doit gagne n-1. Ainsi:
\begin{eqnarray*}
  P(n) & = &P(n-1)\sum_{k=0}(1-p)^kq^kp= P(n-1)\frac{p}{1-(1-p)q}\\
  & = &P(n-1)P(1)
\end{eqnarray*}

Récursivement on a :
\begin{eqnarray*}
  P(n) & = &P(n-1)\sum_{k=0}(1-p)^kq^kp= \frac{p}{1-(1-p)q}\\
  & = &P(n-1)P(1)\\
  & = &P(n-2)P(1)P(1)\\
  & = &P(1)^n
\end{eqnarray*}

%%%%%%%%%%%%%%%%%%%%% 
% Parte B exercício 1
%%%%%%%%%%%%%%%%%%%%% 
\subsection{}

En sachant que B  est au service au départ, la probabilité  que A gagne un point
est, ou il gagne deux points suivantes (qp), ou il gagne le premier (q),perde le
deuxième (p-1) et il gagne le
troisième (q) e il gagne le quatrième (p) et ainsi de suite. Le dessin ci-dessous
représente les états du jeu que sont dans un cycle.

% The sloped option gives rotated edge labels. Personally
% I find sloped labels a bit difficult to read. Remove the sloped options
% to get horizontal labels. 
\begin{tikzpicture}[grow=right, sloped]
  \node[bag] {Service B}
  child {
    node[bag] {Service A} 
    child {
      node[bag,label=
      {Service B}] 
      {}
      child {
        node[bag] {Service A} 
        child {
          node[end,label=
          {Service B}] {}
          edge from parent
          node[above] {1-p}
          node[below]  {}
        }
        child {
          node[bag] {Point 1} 
          edge from parent         
          node[above] {p}
          node[below]  {}
        }
        edge from parent
        node[above] {q}
        node[below]  {}
      }
      edge from parent
      node[above] {1-p}
      node[below]  {}
    }
    child {
      node[bag] {Point 1} 
      edge from parent         
      node[above] {p}
      node[below]  {}
    }
    edge from parent 
    node[above] {q}
    node[below]  {}
  }
  ;
\end{tikzpicture}


Le tableau ci-dessous
représente le probabilité de un point.

\begin{table}[h!]
  \begin{centering}
    \begin{tabular}{|c|c|c|}
      \hline 
      k & nombre services & Probabilité que A gagne le point après n services\tabularnewline
      \hline 
      \hline 
      0 & 2 & $pq$\tabularnewline
      \hline 
      1 & 4 & $(1-p)q²p$\tabularnewline
      \hline 
      2 & 6 & $(1-p)²q³p$\tabularnewline
      \hline 
      3 & 8 & $(1-p)³q^{4}p$\tabularnewline
      \hline 
    \end{tabular}
    \par\end{centering}
  \caption{Probabilite 1 point}
\end{table}

En utilisant le tableau, on peut savoir la $P[n=1]$:

\begin{eqnarray*}
  Q(n=1)  &  =  &  \sum_{k=0}(1-p)^kq^{k+1}p= \frac{pq}{1-(1-p)q}  ;\quad  \text{(suite
    géométrique avec (1-p)q<1) }
\end{eqnarray*}

Pour que A gagne n points, il doit gagne n-1. Ainsi:
\begin{eqnarray*}
  Q(1) & = &P(1)q\\
  Q(n) & = &Q(1)^n\\
  & = &(P(1)p)^n\\
  & = &(P(1)^np^n\\
  & = &P(n)q^n
\end{eqnarray*}

\subsection{}
En sachant que $Q(1)=P(1)q$, et que la probabilité de que A gagne le point si il
est en  service est  p au première  service plus  la probabilité qu'il  perde le
première service fois la probabilité qu'il gagne si B a le service!

\begin{eqnarray*}
  P(1) & = &p+ (1-p)Q(1)\\
  & = &p+ q(1-p)P(1)
\end{eqnarray*}

Ainsi:
\begin{eqnarray*}
  P(1)(1-q+qp)& = &p\\
  P(1)& = &\frac{p}{(1-q+qp)}
\end{eqnarray*}

En récurrence:

\begin{eqnarray*}
  P(n) & =& P(1)^n\\
  P(n)& = &\frac{p^n}{(1-q+qp)^n}
\end{eqnarray*}


\subsection{}
Soit $\frac{1}{2}$ la probabilité que A commence, ainsi comme B commence.
Alors, la probabilité que:

\begin{eqnarray*}
 P[n=21] &=&\frac{P[n=21]}{2}+\frac{Q[n=21]}{2}\\
 &=& \frac{P[1]^{21}}{2}+\frac{q^{21}P[1]^{21}}{2}\\
 &=& \frac{(1+q^{21})P[1]^{21}}{2}\\
 &=& \frac{(1+q^{21})}{2}(\frac{p}{1-q+pq})^{21}\\
\end{eqnarray*}



\section{Exercice 2 }
\subsection{}

Comme la variable aléatoire $X_2$ a de valeurs dans $\mathbb{R}_+$ et $X_1$ dans
$\mathbb{R}$, une composition  entre le tuple $(X_1 , X_2)$  aura ses valeurs dan
$\mathbb{R}\times\mathbb{R}_+$. De  cette façon,  la densité $f(x1,  x2)$ aura
valeur nulle en dehors de $\mathbb{R}\times\mathbb{R}_+$.

Ainsi, pour une fonction $g:\mathbb{R}^2 \rightarrow \mathbb{R}_+$:
\begin{equation}
 \mathbb{E}[g(\frac{X_1}{X_2},X_2)]=\int_{x_1 \in \mathbb{R}} \int_{x_2>0}g(\frac{X_1}{X_2},X_2)f(x_1,x_2)dx_1dx_2
\end{equation}

Si   on  appelle   $a=\frac{x_1}{x_2}$  et   $b=x_2$,  on   a   $x_2(a,b)=b$  et
$x_1(a,b)=ab$.   La deuxième  étape est  calculer le  jacobien du  changement de
variable inverse:

$J=\left[ \begin {array}{cc} {\frac {d}{da}}x1&{\frac {d}{da}}x2\\\noalign{\medskip}{\frac {d}{db}}x1&{\frac {d}{db}}x2\end {array}
\right] = b*1-a*0=b
$ 

%dx1/da=b dx2/da=0
%dx1/db=a dx2/db=1

Par conséquent:
\begin{equation}
 \mathbb{E}[g(\frac{X_1}{X_2},X_2)]=\int_{a            \in           \mathbb{R}}
 \int_{b>0}g(a,b)f(ab,b)bdadb=\int \int_{\mathbb{R}^2} \mathds{1}_\mathbb{R}(u)\mathds{1}_{\mathbb{R}_+}(v)g(a,b)f(ab,b)bdadb
\end{equation}

Donc, la densité de $U(\frac{X_1}{X_2},X_2)$ est $f(a,b)=\mathds{1}_\mathbb{R}(a)\mathds{1}_{\mathbb{R}_+}(a)f(ab,b)b$

\subsection{}

Comme $Y=\frac{X_1}{X_2}$, on a $U(Y,X_2)$, ainsi,  pour trouver la loi de Y, on
calcule la densité marginale de la première composante de U:

\begin{equation}
f_Y(a)=\int_0^\infty f_1(ab,b)bdb ; \quad a \in \mathbb{R}
\end{equation}


\section{Exercice 3}

\subsection{}
Si $X_1$ e $X_2$ sont deux  v.a. de Poisson indépendantes, de paramètres $\mu_1$
et $\mu_2$. La loi caractéristique est donné par:
\begin{equation}
  \phi_x(t)=e^{\mu(e^{it}-1)}
\end{equation}

Comme $X_1$ e $X_2$ sont indépendantes:
\begin{equation}
  \phi_{x_1+x+2}(t)=\phi_{x_1}(t)\phi_{x_2}(t)=e^{\mu_1(e^{it}-1)}e^{\mu_2(e^{it}-1)}=
  e^{(\mu_1+\mu_2)(e^{it}-1)}
\end{equation}

Ainsi la somme  de $X_1$ e $X_2$(paramètres $\mu_1$ et $\mu_2$)  suit une loi de
Poisson de paramètre $\mu=\mu_1+\mu_2$.

\subsection{}
Pour calculer la espérance et la  variance d'une loi de poisso, on doit utiliser
génératrice.

\begin{equation}
  g_x(t)=\sum_{n=0}^{\infty}z^nP[X=n]=E[z^X]
\end{equation}

Ainsi:
\begin{eqnarray*}
  g_x(t)&=& \sum_{n=0}^{\infty}z^nP[X=n]\\
  &=& e^{-\lambda}\sum_{n=0}^{\infty}\frac{z^n(z\lambda)^k}{k!}\\
  &=& e^{-\lambda}e^{\lambda z}\\
  &=& e^{\lambda(z-1)}
\end{eqnarray*}
\begin{eqnarray*}
  g'_x(t)&=& \lambda ze^{\lambda(z-1)}\\
  g'_x(1)&=&E[X]= \lambda\\
  g^{''}_x(t)&=& \lambda e^{\lambda(z-1)}+(\lambda z)^2e^{\lambda(z-1)}\\
  g^{''}_x(1)&=& \lambda + \lambda^2 = E[X^2]\\
  Var[X] &=& E[X^2]-E[X]^2 = \lambda
\end{eqnarray*}

Si le paramètre est 1, alors:
\begin{eqnarray*}
  E[X]&=& 1\\
  Var[X] &=& 1
\end{eqnarray*}

\subsection{}

On va considérer que $P_1,...,P_n$ sont indépendantes et suivent une loi de poisson de paramètre 1, donné par:
\begin{eqnarray*}
  e^{-\lambda}\frac{\lambda^k}{k!}=P[k]\\
\end{eqnarray*}

Ainsi, on a :
\begin{eqnarray*}
  e^{-n}\sum_{k=0}^n\frac{n^k}{k!}&=&\mathbb{P}[P_1+P_2+...+P_n\leq n]\\
\end{eqnarray*}

Si on appelle $S_n=P_1+\dots+P_n$, le valeur de la moyenne ira être $n\mu$ et la
variance  $n\sigma^2$. En  appellent $Z_n=\frac{S_n-n\mu}{\sigma  \sqrt{n}}$, on
peut  utiliser  la  loi  de  limite  central pour  approcher  $Z_n$  a  une  loi
normal.      Soit,       comme      considéré      avant,$\mu=\sigma=1$,      on
a,$Z_n=\frac{S_n-n}{\sqrt{n}}$. Donc, on peut récrire la équation suivante
\begin{eqnarray*}
 \mathbb{P}[P_1+P_2+...+P_n\leq n]&=& \mathbb{P}[P_1+P_2+...+P_n-n \leq 0] \\
 &=& \mathbb{P}[\frac{P_1+P_2+...+P_n-n}{\sqrt{n}} \leq 0]  ; \quad \text{n > 0};
\end{eqnarray*}

Pour terminer, par le théorème central limité(TCL), on a:

\begin{eqnarray*}
 \lim_{n\rightarrow\infty} e^{-n}\sum_{k=0}^n\frac{n^k}{k!}=\mathbb{P}(N\leq 0)
\end{eqnarray*}

On  peut   vérifier  para  le  graphique   au-dessous,  que  si   on  centre  la
distribution($\mu=0$),la moitie de  la courbe va entre à gauche  de 0, cela nous
donné la probabilité de $\frac{1}{2}$.

\begin{center}
\begin{tikzpicture} 
  \begin{axis}
    [separate axis lines, % important !
    xtick={0,-5,5,-10,10,-15,15},
    xticklabels={$\mu$,$-\sigma$,$\sigma$,$-2\sigma$,$2\sigma$,$-3\sigma$,$3\sigma$},
    ytick={0,0.25,0.5,0.75,1.0},
    yticklabels={0,0.1,0.2,0.3,0.4},
    legend style={at={(0.5,-0.15)}, anchor=north,legend columns=-1},
    ]
    \addplot[blue,id=DoG, samples=100,
    domain=-15:15] gnuplot{exp(-x*x*0.03/2.66)};
    \legend{Normal distribution};
  \end{axis} 
\end{tikzpicture}
\end{center}

Conclusion:

\begin{eqnarray*}
 \lim_{n\rightarrow\infty} e^{-n}\sum_{k=0}^n\frac{n^k}{k!}=\mathbb{P}(N\leq 0)=\frac{1}{2}
\end{eqnarray*}

\end{document}