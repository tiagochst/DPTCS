%-%-%-%-%-%-%-%-%-%-%-%-%-%-%-%-%-%-%-%-%-%-%-%-%-%
% MDI200: Probabilités                            %  
% Devoir 1                                        %
% Data:05/10/2011                                 %
% Paris,France                                    % 
% Groupe:                                         %
%       - Tiago Chedraoui Silva                   % 
%-%-%-%-%-%-%-%-%-%-%-%-%-%-%-%-%-%-%-%-%-%-%-%-%-%

\documentclass[a4paper]{article}

%%% fontes %%%
\usepackage[french]{babel}
\usepackage[T1]{fontenc}
\usepackage[utf8]{inputenc}   % acentuação
\usepackage{ae,aecompl,aeguill}  % pdfs plus beaux =)
\usepackage{dsfont}

%%% autres %%%
\usepackage{multirow}
\usepackage{textcomp}
\usepackage{color}       
\usepackage{indentfirst}     
\usepackage{multicol}   
\usepackage[linkbordercolor={1 1 1},urlcolor=black,colorlinks=true]{hyperref} % links
\usepackage{subfig}
\usepackage[letterpaper]{geometry}
\geometry{verbose,lmargin=3cm,rmargin=3cm}

\usepackage{amsfonts}
\usepackage{amsmath} 

%\setcounter{tocdepth}{0}

\renewcommand{\thefigure}{\arabic{figure}}

\date{Octobre 17, 2011}
% Capa estilizada %
\newcommand*{\titleTMB}{\begingroup \centering \settowidth{\unitlength}{\LARGE EE531} {\large\scshape MDI200 - Cycle d'harmonisation }\\[0.2\baselineskip] \rule{11.0cm}{1.6pt}\vspace*{-\baselineskip}\vspace*{2pt} \rule{11.0cm}{0.4pt}\\[\baselineskip] {\LARGE Devoir 02}\\\vspace*{\baselineskip}  {\itshape Introduction aux Probabilités - Deuxième Semestre de 2011}\\ \rule{11.0cm}{0.4pt}\vspace*{-\baselineskip}\vspace{3.2pt} \rule{11.0cm}{1.6pt}\\[\baselineskip] {\large\scshape Professeur: Michel Grojnowski}\par \vfill {\normalsize   \scshape 
    \begin{center} 
      \begin{tabular}{  l  l  p{5cm} } 
        Tiago Chedraoui Silva  & Cassier: 214\\
      \end{tabular} \end{center}
    \itshape 17 octobre de 2011    }\\[\baselineskip] \vspace{3.2pt} \endgroup}


\begin{document}
\titleTMB 
\newpage

\section{Exercice 1}
\subsection{}
\subsection{}
\subsection{}
\subsection{}
\section{Exercice 2 }
\subsection{}
\subsection{}
\section{Exercice 3}

Si $X_1$ e $X_2$ sont deux  v.a. de Poisson indépendantes, de paramètres $\mu_1$
et $\mu_2$. La loi caractéristique est donné par:
\begin{equation}
\phi_x(t)=e^{\mu(e^{it}-1)}
\end{equation}

Comme $X_1$ e $X_2$ sont indépendantes:
\begin{equation}
\phi_{x_1+x+2}(t)=\phi_{x_1}(t)\phi_{x_2}(t)=e^{\mu_1(e^{it}-1)}e^{\mu_2(e^{it}-1)}=
e^{(\mu_1+\mu_2)(e^{it}-1)}
\end{equation}

Ainsi la somme  de $X_1$ e $X_2$(paramètres $\mu_1$ et $\mu_2$)  suit une loi de
Poisson de paramètre $\mu=\mu_1+\mu_2$.

\subsection{}
Pour calculer la espérance et la  variance d'une loi de poisso, on doit utiliser
génératrice.

\begin{equation}
  g_x(t)=\sum_{n=0}^{\infty}z^nP[X=n]=E[z^X]
\end{equation}

Ainsi:
\begin{eqnarray*}
 g_x(t)&=& \sum_{n=0}^{\infty}z^nP[X=n]\\
 &=& e^{-\lambda}\sum_{n=0}^{\infty}\frac{z^n(z\lambda)^k}{k!}\\
 &=& e^{-\lambda}e^{\lambda z}\\
&=& e^{\lambda(z-1)}
\end{eqnarray*}
\begin{eqnarray*}
 g'_x(t)&=& \lambda ze^{\lambda(z-1)}\\
 g'_x(1)&=&E[X]= \lambda\\
 g^{''}_x(t)&=& \lambda e^{\lambda(z-1)}+(\lambda z)^2e^{\lambda(z-1)}\\
 g^{''}_x(1)&=& \lambda + \lambda^2 = E[X²]-E[X]\\
Var[X] &=& E[X²]-E[X]² = \lambda^2
\end{eqnarray*}

Si le paramètre est 1, alors:
\begin{eqnarray*}
E[X]&=& 1\\
Var[X] &=& 1
\end{eqnarray*}


\subsection{}
\subsection{}


\end{document}