% -%-%-%-%-%-%-%-%-%-%-%-%-%-%-%-%-%-%-%-%-%-%-%-%-%
% Devoir 01 FLE                                    %  
% Data:05/10/2011                                  %
% Paris,France                                     %
% - Tiago Chedraoui Silva                          %
% -%-%-%-%-%-%-%-%-%-%-%-%-%-%-%-%-%-%-%-%-%-%-%-%-%
\documentclass[dvips,11pt,xcolor=dvipsnames]{beamer}

%%% fontes %%%
\usepackage[french]{babel}
\usepackage[T1]{fontenc}
\usepackage[utf8]{inputenc}   
\usepackage{ae,aecompl,aeguill}  % pdfs plus beaux =)

%%% matematicos %%%
\usepackage{amsmath}
\usepackage{amssymb}
\usepackage{mathptmx}

%%% figuras %%%
\usepackage{graphicx}
\usepackage{wrapfig}

%%% tabelas %%%
\usepackage{colortbl}
\usepackage{array}
\usepackage{longtable}
\usepackage{fancyvrb}
\usepackage{color}

%%% outros %%%
\usepackage{url}
\usepackage{textcomp}
\usepackage{hyperref} %internal links
\usepackage{color}       
\usepackage{indentfirst} %retira padrao americano de paragrafos
\usepackage{multicol}    
\numberwithin{table}{section}
\numberwithin{figure}{section} %numercao de figuras por secao

%%% extras %%%
\RequirePackage{marvosym} % figuras \Letter \Email 
\usepackage{fancyhdr}     % Headers
\usepackage{epsf}
\usepackage{tikz}
\usepackage{mhchem}    
\usetikzlibrary{arrows}
\tikzstyle{block}=[draw opacity=0.7,line width=1.4cm]

%%% Beamer style %%%
\usetheme{Warsaw}
%\usecolortheme[named=Green]{structure}
\usefonttheme[onlylarge]{structurebold} % fontes em negrito
\setbeamerfont*{frametitle}{size=\normalsize,series=\bfseries}%define tamanhos de letras
\setbeamertemplate{navigation symbols}{}% barra de navegacao superior
% \setbeamercovered{transparent}


% -%-%-%-%-%-%-%-%-%
% Inicio Slides  %
% -%-%-%-%-%-%-%-%-%

\title{FLE}                               
% Refletindo os valores humanos na era digital
\subtitle{Resources FLE}                               
\author[Tiago S.]{
  Tiago Chedraoui Silva \\
}
\institute{}

\date{Octobre 17, 2011}
%\date{\today}

\begin{document}

\begin{frame}
  \titlepage
\end{frame}

%\begin{frame}{Outline}
%  \tableofcontents
%\end{frame}


\section{Ressources FLE}

\begin{frame}{Outils}
 
 \begin{block}{Connaissances grammaticales}
    \begin{itemize}
    \item  CCMD
    \item Le point du FLE
    \end{itemize}
  \end{block}

 \begin{block}{Vocabulaire}
    \begin{itemize}
    \item Le monde
    \end{itemize}
  \end{block}

 \begin{block}{Écrire}
    \begin{itemize}
    \item CCDMD
    \end{itemize}
  \end{block}

 \begin{block}{Culture Française}
    \begin{itemize}
    \item Arte 
    \end{itemize}
  \end{block}

 \begin{block}{Litérature Française}
    \begin{itemize}
    \item Le monde
    \item LCI
    \end{itemize}
  \end{block}

\end{frame}


%\begin{frame}{Outils}

%\end{frame}

\begin{frame}{Site intéressant}
 \begin{block}{Centre collégial de développement de matériel didactique (CCDMD)}
    \begin{itemize}
    \item \url{http://www.ccdmd.qc.ca/}.
    \item \url{http://www.tv5.org/}.
    \end{itemize}
  \end{block}
\end{frame}

%\section{Vocabulaire}
%\section{Écrire}
%\section{Culture Française}
%\section{Littérature Française}
%\section{Site intéressant}
\end{document}
