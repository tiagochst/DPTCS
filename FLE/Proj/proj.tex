% -%-%-%-%-%-%-%-%-%-%-%-%-%-%-%-%-%-%-%-%-%-%-%-%-%
% FLE % 
% Data:28/11/2011                                 %
% Paris,France                                    % 
% Groupe:                                         %
% - Tiago Chedraoui Silva                   % 
% - Angie Anazgo la Rosa %
% -%-%-%-%-%-%-%-%-%-%-%-%-%-%-%-%-%-%-%-%-%-%-%-%-%

\documentclass[a4paper,11pt]{article}

\usepackage[francais,listings,algo]{tcs}

% Cover %
\def \ttprofname{Isabelle Lallemand} % teachers name
\def \ttabrv{FLE804 } % abbreviation of names class
\def \ttabrvxt{} % period
\def \mytitle{Proposition de projet } % Big title
\def \mysubtitle{FLE Avancé - Deuxième semestre de 2011} % subtitle
\def \ttauthi{Angie Añazgo La Rosa} % author's name
\def \ttxti{Casier: 200 } % Extra text right side of name
\def \ttauthii{Tiago Chedraoui Silva} % author's name
\def \ttxtii{Casier: 214 } % Extra text right side of name
\def \ttdate{Novembre 15, 2011} % date

\begin{document}
\titleTMB 
\newpage
\tableofcontents
\newpage

\section{Introduction}

\subsection{Origine du débat}
\subsection{Sources utilisés}
Quels sont les sources de notre travail? Une liste de sources.
\subsection{Motivation}
Notre motivation pour travailler ce sujet
\subsection{Description du sujet}
Décrire les impacts du changement climatique, la cause, etc.


\section{Points de vue}
\subsection{Réaction de la société}
On décrira chaque position: les écologistes, le gouvernement, la société en général, la presse, etc
Il faut choisir et résumer lesquels

\subsection{Contraintes}

\subsubsection{D'ordre scientifique}

\begin{itemize}
\item Pression des politiciens sur les scientifiques
\item Pression des écologistes sur les politiciens
\end{itemize}

\subsubsection{Sociales}

\begin{itemize}
\item Médiatisation sensationnaliste
\item Société de consommation
\end{itemize}

\subsubsection{Gouvernementales}

\begin{itemize}
\item Protocole de Kyoto
\item Taxes sur les émissions de Gaz à effet de serre
\end{itemize}

\subsubsection{Économiques}

\begin{itemize}
\item Nécessité de maintenir la croissance économique
\item Coûts associés aux changements technologiques
\end{itemize}


\section{Débat: Solutions proposés}


\section{Sources}
Ici on liste d'où on a obtenu l'information

\subsection*{Plan}
\label{subsec:cronograma}
Le développement du projet de recherche est liées aux activités suivants. Le
calendrier est présenté au tableau~\ref{tab:cronograma} et est organisée dans un
période de 3 mois, avec début dans novembre de 2011 et fin en février de 2012.

\begin{enumerate}
\item Présentation et plan du travail de recherche.
\item Recherche bibliographique sur le sujet.
\item Production du dossier.
\item Production de diapositives.
\item Simulation de la présentation.
\item Révision du diapositives.
\item Exposé.

\end{enumerate}

\begin{table}[h]
  \caption{Calendrier des activités}
  \begin{center} {
      \begin{tabular}{ |c|c|c|c|c|c|c|c|c|c|}
        \hline
        & \multicolumn{4}{|c|}{2011} & \multicolumn{4}{|c|}{2012}\\ \hline
        \hline
        &   \multicolumn{2}{|c|}{Novembre}   &  \multicolumn{2}{|c|}{Décembre}& \multicolumn{2}{|c|}{Janvier} & \multicolumn{2}{|c|}{Février} \\ \hline
        \hline
        & 1-15 & 16-30 & 1-15 & 16-30 & 1-15 & 16-30 & 1-15 & 16-30 \\ \hline
        1 & $\bullet$ & & & & & & & \\ \hline
        2 & $\bullet$ & $\bullet$ & $\bullet$ & & & & & \\ \hline
        3 &  &  & $\bullet$ & $\bullet$& & & & \\ \hline
        4 & & & & & $\bullet$ & $\bullet$& $\bullet$ & \\ \hline
        5 & & & & & & & $\bullet$ & \\ \hline
        6 & & & & & & & $\bullet$ & \\ \hline
        7 & & & & & & & & $\bullet$\\ \hline

      \end{tabular}
    }\end{center}
  \label{tab:cronograma}
\end{table}


\end{document}