%% start of file `letter.tex'.
%% Copyright 2006-2010 Xavier Danaux (xdanaux@gmail.com).
%
% This work may be distributed and/or modified under the
% conditions of the LaTeX Project Public License version 1.3c,
% available at http://www.latex-project.org/lppl/.


\documentclass[11pt]{article}

\usepackage[french]{babel}
\usepackage[T1]{fontenc}
\usepackage[utf8]{inputenc}   
\usepackage{ae,aecompl,aeguill}  % pdfs plus beaux =)
\usepackage{lmodern}
\usepackage{marvosym}
\usepackage{color}
\usepackage{ifpdf}
\ifpdf
  \usepackage[pdftex]{graphicx}
\else
  \usepackage[dvips]{graphicx}\fi

\pagestyle{empty}

\usepackage[scale=0.8]{geometry}
\setlength{\parindent}{0pt}
\addtolength{\parskip}{6pt}

\def\firstname{Tiago}
\def\familyname{Chedraoui Silva}
\def\FileAuthor{\firstname \familyname}
\def\FileTitle{\firstname \familyname's cover letter}
\def\FileSubject{Cover letter}
\def\FileKeyWords{\firstname \familyname, Cover letter}

\usepackage{url}
\renewcommand{\ttdefault}{pcr}
\urlstyle{tt}
\ifpdf
  \usepackage[pdftex,pdfborder=0,breaklinks,baseurl=http://,pdfpagemode=None,pdfstartview=XYZ,pdfstartpage=1]{hyperref}
  \hypersetup{
    pdfauthor   = \FileAuthor,%
    pdftitle    = \FileTitle,%
    pdfsubject  = \FileSubject,%
    pdfkeywords = \FileKeyWords,%
    pdfcreator  = \LaTeX,%
    pdfproducer = \LaTeX}
\else
  \usepackage[dvips]{hyperref}
\fi

\renewcommand{\familydefault}{\sfdefault}% for use with a résumé using sans serif fonts;
%\renewcommand{\familydefault}{\rmdefault}% for use with a résumé using sans serif fonts;
\newcommand{\red}[1]{{\color{red}{#1}}}

\begin{document}
\hfill%
\begin{minipage}[t]{.6\textwidth}
\raggedleft%
{\bfseries Tiago Silva}\\[.35ex]
\small\itshape%
1, boulevard Jourdan, Maison du Maroc\\
75014,Paris,France\\[.35ex]
\Letter~tiagochst@gmail.com
\end{minipage}\\[1em]
%
\begin{minipage}[t]{.4\textwidth}
\raggedright%
{\bfseries Devoir FLE}\\[.35ex]
\small\itshape%
%Lettre à un ami\\
%Lettre à un professeur\\
Lettre à un emplyé administratif\\
\end{minipage}
\hfill % US style
%\\[1em] % UK style
\begin{minipage}[t]{.4\textwidth}
\raggedleft % US style
Le \today
%April 6, 2006 % US informal style
%05/04/2006 % UK formal style
\end{minipage}\\[2em]
\raggedright

% Soutenu
% -- Monsieur le direteur, 
% -- Madame, Monsieur,
% -- Messieurs,

% Standard
% -- Bonjour,
% -- Bonjour Madame,
% -- Cher directeur,
% -- Cher Pierre,
% -- Bonjour à tous,
%

% Familier
% -- Salut,
% -- Hello,
% -- Coucou,
% -- Salut tout le monde,


Madame la directrice ,\\[1.5em]
%
%Yours sincerely,\\[2em] % if the opening is "Dear Mr(s) Doe,"

% Soutenu
% -- Veuillez agréer, M----, l'expression de mes salutations distingués 

% Standard
% -- Très cordialement,
% -- Bien,
% -- Amicalement,
% -- Bonne journée,
% -- À bientôt,
%

% Familier
% -- Bisous,
% -- Bises,
% -- Je t'embrasse,
% -- A plus (A +)
% -- A toute alors,
% -- A toute de suite


Je voudrais  \red{poser} ma candidature  au Comité de l'Enseignement  de Télécom
Paristech  pour  l’année 2011-2012.  Étant  donné  que  je suis  un  étranger,je
pourrais  donner,  pour  les  discussions  sur  les  règlements  scolaires,  une
connaissance internationale, \red{qui l'enrichirait}.

 D'autre  \red{part}, \red{ce}  Comité  examine  également les  cas  des élèves  en
 difficultés lorsque  c'est nécessaire.  Je suis sûr  que les étrangers  sont en
 difficultés, parce qu'une  adaptation à une nouvelle culture,  \red{ne se} fait
 pas  rapidement. En  s'appuyant sur  \red{ce fait},  je pense  que  ma présence
 est indispensable pour que les droits des étrangers \red{soient respectés}.

Veuillez agréer, Madame la directrice, l'expression de mes salutations \red{distinguées},\\[2em]

 % if the opening is "Dear Sir or Madam,"
%
%\includegraphics[scale=0.75]{signature_blue}\\
{\bfseries Tiago Chedraoui Silva}\\
%
\vfill%
%{\slshape Tiago Chedraoui Silva}
%{\slshape Attachment: curriculum vit\ae{}}
\end{document}
